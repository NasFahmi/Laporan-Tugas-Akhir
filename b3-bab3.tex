%==================================================================
% Ini adalah bab 3
% Silahkan edit sesuai kebutuhan, baik menambah atau mengurangi \section, \subsection
%==================================================================

\chapter[METODOLOGI PENELITIAN]{\\ METODOLOGI PENELITIAN}
\section{Waktu dan Jadwal penelitian}
\subsection{Waktu Pelaksanaan Penelitian}
Waktu pelaksanaan penelitian ini direncanakan selama 6 bulan, yaitu dimulai pada bulan Agustus 2025 dan berakhir pada bulan Januari 2026. Rentang waktu tersebut dipilih untuk memastikan seluruh tahapan penelitian dapat dilaksanakan secara sistematis dan terstruktur, mulai dari analisis kebutuhan, perancangan sistem, implementasi, hingga pengujian. Pembagian waktu penelitian disusun secara bertahap agar setiap aktivitas penelitian dapat dilakukan secara optimal sesuai dengan metode yang digunakan, serta memberikan ruang untuk penyesuaian apabila ditemukan kendala selama proses pengembangan sistem.
% tambahkan beberapa kalimat agar tidak 1 paragarf 1 kalimat
\subsection{Jadwal Kegiatan Penelitian}
Berikut adalah serangkaian jadwal kegiatan yang dilakukan selama pelaksanaan penelitian ini, yang diuraikan pada Tabel \ref{tab:jadwal_pelaksanaan}.
% \vspace{-.3cm}
\setlength{\arrayrulewidth}{0.8pt}
\renewcommand{\arraystretch}{1.4}
\setlength{\tabcolsep}{6pt}

\setlength{\arrayrulewidth}{0.8pt}
\renewcommand{\arraystretch}{1.4}
\setlength{\tabcolsep}{6pt}
\begin{longtable}{|
	c
	|p{0.22\linewidth}
	|>{\centering\arraybackslash}p{0.08\linewidth}
	|>{\centering\arraybackslash}p{0.08\linewidth}
	|>{\centering\arraybackslash}p{0.08\linewidth}
	|>{\centering\arraybackslash}p{0.08\linewidth}
	|>{\centering\arraybackslash}p{0.08\linewidth}
	|>{\centering\arraybackslash}p{0.08\linewidth}|
}
\caption{Jadwal Pelaksanaan Penelitian}
\label{tab:jadwal_pelaksanaan} \\
\hline
\multirow{3}{*}{\textbf{No}} &
\multirow{3}{*}{\textbf{Nama Kegiatan}} &
\multicolumn{6}{c|}{\textbf{Waktu Pelaksanaan}} \\ \cline{3-8}

& &
\multicolumn{5}{c|}{\textbf{2025}} &
\multicolumn{1}{c|}{\textbf{2026}} \\ \cline{3-8}

& &
\textbf{Ags} &
\textbf{Sep} &
\textbf{Okt} &
\textbf{Nov} &
\textbf{Des} &
\textbf{Jan} \\ \hline
\endfirsthead

\hline
\multirow{3}{*}{\textbf{No}} &
\multirow{3}{*}{\textbf{Nama Kegiatan}} &
\multicolumn{6}{c|}{\textbf{Waktu Pelaksanaan}} \\ \cline{3-8}

& &
\multicolumn{5}{c|}{\textbf{2025}} &
\multicolumn{1}{c|}{\textbf{2026}} \\ \cline{3-8}

& &
\textbf{Ags} &
\textbf{Sep} &
\textbf{Okt} &
\textbf{Nov} &
\textbf{Des} &
\textbf{Jan} \\ \hline
\endhead

\hline
\endfoot

\hline
\endlastfoot

1 & Analisis Kebutuhan Sistem
& \cellcolor{yellow} & & & & & \\ \hline

2 & Perumusan Ide dan Studi Awal Permasalahan
& \cellcolor{yellow} & & & & & \\ \hline

3 & Penyusunan Proposal dan Konsep Solusi
& \cellcolor{yellow} & & & & & \\ \hline

4 & Pengumpulan Proposal dan Video Presentasi
& & \cellcolor{yellow} & & & & \\ \hline

5 & Seleksi Proposal dan Penetapan Finalis
& & \cellcolor{yellow} & & & & \\ \hline

6 & Presentasi dan Penjurian Akhir
& & \cellcolor{yellow} & & & & \\ \hline

7 & Pengembangan Arsitektur \textit{Frontend} dan \textit{Client Data Layer}
& & & \cellcolor{yellow} & \cellcolor{yellow} & \cellcolor{yellow} & \\ \hline

8 & Implementasi Dashboard dan TanStack Query
& & & \cellcolor{yellow} & \cellcolor{yellow} & \cellcolor{yellow} & \\ \hline

9 & Pengujian Sistem (Black-box Testing)
& & & & \cellcolor{yellow} & \cellcolor{yellow} & \cellcolor{yellow} \\ \hline

10 & Penyusunan Laporan Tugas Akhir
& & & & & \cellcolor{yellow} & \cellcolor{yellow} \\ \hline

\end{longtable}

\FloatBarrier

% ini nanti perllu di tambah kan ngapain saja di setiap bulan dan kegiatan
% \vspace{-.3cm}
Berdasarkan jadwal kegiatan penelitian yang telah disusun, pelaksanaan penelitian diawali pada bulan Agustus 2025 dengan tahap analisis kebutuhan sistem. Pada tahap ini dilakukan identifikasi permasalahan, pengumpulan kebutuhan pengguna, serta menyusun kebutuhan terhadap data dan API yang digunakan sebagai sumber data dashboard analisis sentimen. Tahap analisis ini menjadi dasar bagi tahapan penelitian selanjutnya.
Pada bulan September 2025, kegiatan penelitian difokuskan pada perancangan sistem, yang meliputi perancangan arsitektur \textit{Frontend}, perancangan \textit{Client Data Layer}, serta perancangan antarmuka pengguna (UI). Tahap perancangan bertujuan untuk menghasilkan rancangan sistem yang terstruktur dan sesuai dengan kebutuhan yang telah dianalisis sebelumnya.
Pada periode bulan September hingga Desember 2025, kegiatan penelitian difokuskan pada tahapan perancangan dan implementasi sistem yang dilakukan secara iteratif dan saling tumpang tindih. Pada periode ini, proses perancangan antarmuka pengguna (desain antarmuka), perancangan sistem, perancangan arsitektur \textit{Frontend} dan \textit{Client Data Layer}, serta implementasi \textit{Frontend} menggunakan TanStack Query tidak dilakukan secara terpisah dan linier, melainkan berjalan secara bersamaan sesuai dengan karakteristik metode Fountain yang fleksibel.
Pendekatan ini memungkinkan hasil dari tahap implementasi \textit{Frontend} untuk secara langsung digunakan sebagai dasar penyesuaian pada tahap perancangan sistem maupun perancangan arsitektur \textit{Frontend}. Sebaliknya, perubahan pada desain antarmuka atau struktur arsitektur juga dapat segera diimplementasikan dan diuji tanpa harus menunggu selesainya seluruh tahapan sebelumnya. Dengan demikian, proses pengembangan sistem dapat berjalan lebih adaptif terhadap kebutuhan dan temuan selama penelitian.
Pelaksanaan tahapan-tahapan tersebut secara paralel bertujuan untuk menjaga konsistensi antara rancangan arsitektur, mekanisme pengelolaan data pada \textit{Client Data Layer}, serta implementasi antarmuka pengguna. Pola kerja ini sejalan dengan prinsip metode Fountain yang memungkinkan terjadinya pengulangan dan penyesuaian antar-tahapan pengembangan tanpa mengganggu keseluruhan alur penelitian. 
Selain itu, pendekatan paralel ini memberikan ruang bagi peneliti untuk melakukan evaluasi teknis secara berkelanjutan terhadap perilaku pengelolaan data \textit{Frontend}, khususnya terkait konsistensi data dan pengendalian pemanggilan API. Dengan adanya evaluasi berulang tersebut, setiap penyesuaian yang dilakukan dapat langsung divalidasi kesesuainya terhadap tujuan penelitian.

\section{Metode Fountain}
Berdasarkan landasan teori dan pembahasan metode penelitian yang telah diuraikan pada Bab II, penelitian ini menggunakan metode Fountain sebagai pendekatan dalam pengembangan sistem. Metode Fountain dipilih karena memiliki karakteristik fleksibel dan iteratif, sehingga memungkinkan tahapan analisis, perancangan, implementasi, dan pengujian dilakukan secara saling tumpang tindih sesuai dengan kebutuhan penelitian. Karakteristik tersebut dinilai sesuai dengan pengembangan aplikasi \textit{frontend} yang bersifat data-driven dan memerlukan evaluasi berulang terhadap arsitektur dan pengelolaan data.
Pada Bab III ini, metode Fountain diterapkan secara sistematis pada penelitian yang dilakukan. Pembahasan difokuskan pada tahapan penerapan metode Fountain dalam konteks pengembangan Dashboard Analisis Sentimen, mulai dari tahap analisis hingga tahap testing. Uraian pada setiap tahapan menjelaskan aktivitas yang dilakukan dalam penelitian ini tanpa mengulang pembahasan teoritis yang telah disampaikan pada Bab II.
Pemilihan metode Software Development Life Cycle (SDLC) yang tepat merupakan langkah strategis yang krusial untuk memastikan keberhasilan proyek serta mencegah pembengkakan biaya dan waktu pengembangan \citep{Aniley2024}. Berdasarkan pertimbangan tersebut, penelitian ini mengadopsi metode Fountain yang dinilai relevan karena memiliki karakteristik fleksibel namun tetap terstruktur. Efektivitas metode ini didukung oleh penelitian \citep{Sastra2023} pada perancangan sistem SIP-PTK, yang menunjukkan bahwa model Fountain mampu memandu tahapan analisis, desain, implementasi, hingga pengujian secara efektif pada sistem yang memiliki kebutuhan pengolahan data yang spesifik.
Relevansi metode Fountain juga sejalan dengan fokus penelitian ini, yaitu pengembangan aplikasi \textit{frontend} yang bersifat data-driven dan memiliki ketergantungan tinggi terhadap interaksi antar-komponen. Dalam pengembangan \textit{frontend}, tahapan perancangan arsitektur, implementasi, dan pengujian tidak berjalan secara linier, melainkan saling berkaitan dan sering memerlukan penyesuaian berdasarkan hasil evaluasi sistem. Penerapan metode Fountain memungkinkan tahapan pengembangan dan testing berjalan secara paralel serta saling tumpang tindih (overlapping) tanpa menghilangkan struktur dan urutan penelitian yang jelas.
Melalui pendekatan ini, aktivitas analisis kebutuhan, perancangan arsitektur \textit{frontend}, implementasi \textit{Client Data Layer}, serta pengujian sistem dapat dilakukan secara berulang dan saling mempengaruhi. Hasil pada satu tahapan dapat langsung digunakan sebagai dasar penyesuaian pada tahapan lainnya, seperti perubahan struktur data atau mekanisme sinkronisasi, tanpa harus menunggu seluruh siklus pengembangan selesai. Dengan demikian, proses pengembangan sistem menjadi lebih adaptif dan terkontrol, sehingga diharapkan mampu menghasilkan hasil penelitian yang sesuai dengan tujuan penelitian secara optimal.
Tahapan penerapan metode fountain dalam penelitian ini terdiri atas beberapa tahap sebagai berikut:
\subsection{Analysis}
  Pada tahap analisis, penelitian ini mengkaji kebutuhan dan permasalahan pada pengembangan aplikasi \textit{frontend} berbasis web yang berfungsi sebagai dashboard analisis sentimen UMKM. Analisis dilakukan terhadap kebutuhan penyajian hasil analisis sentimen media sosial dalam bentuk visualisasi yang informatif, ringkas, dan mudah dipahami oleh pengguna. Informasi yang dianalisis mencakup kebutuhan penampilan ringkasan sentimen, distribusi sentimen, serta indikator lain yang relevan untuk memantau persepsi konsumen secara umum.
  Analisis juga difokuskan pada alur pengelolaan data antara backend dan \textit{frontend}. Pada penelitian ini, proses pengumpulan data media sosial dan analisis sentimen sepenuhnya dilakukan di sisi backend, sementara \textit{frontend} bertugas mengonsumsi data hasil analisis melalui REST API. Kondisi ini menuntut adanya mekanisme pengelolaan data \textit{frontend} yang mampu menerima data secara konsisten dan menyajikannya ke berbagai komponen dashboard tanpa menimbulkan inkonsistensi informasi.
  Selain itu, pada tahap analisis diidentifikasi permasalahan pengelolaan data \textit{frontend} pada aplikasi dashboard yang bersifat data-driven, khususnya ketika data yang sama digunakan oleh banyak komponen antarmuka secara bersamaan. Permasalahan yang dianalisis meliputi potensi terjadinya permintaan data berulang, kesulitan sinkronisasi data antar-komponen, serta kebutuhan pembaruan data secara terkontrol. Hasil analisis ini menjadi dasar dalam menentukan kebutuhan arsitektur \textit{frontend} yang lebih terstruktur dan efisien.
  Pengguna sistem dalam penelitian ini dianalisis sebagai pengguna umum atau pelaku UMKM yang memanfaatkan dashboard untuk memantau informasi sentimen. Aktivitas pengguna dibatasi pada pengamatan dan eksplorasi informasi yang ditampilkan, tanpa keterlibatan langsung dalam proses pengolahan data. Dengan karakteristik pengguna tersebut, analisis sistem difokuskan pada aspek penyajian informasi dan pengelolaan data di sisi \textit{frontend} agar sesuai dengan tujuan penelitian.
\subsection{Requirement Specification}
  Tahap \textit{requirement specification} bertujuan untuk merumuskan kebutuhan sistem secara terstruktur berdasarkan hasil analisis yang telah dilakukan pada tahap sebelumnya. Pada penelitian ini, spesifikasi kebutuhan difokuskan pada sisi \textit{frontend} dashboard analisis sentimen UMKM, dengan mempertimbangkan karakteristik sistem yang bersifat data-driven dan bergantung pada data hasil analisis sentimen dari backend.
  Kebutuhan fungsional sistem mencakup kemampuan aplikasi \textit{frontend} untuk mengonsumsi data hasil analisis sentimen yang disediakan melalui REST API dan menyajikannya dalam berbagai komponen dashboard. Sistem harus mampu menampilkan ringkasan sentimen, distribusi sentimen, serta indikator pendukung lainnya secara konsisten pada seluruh komponen antarmuka. Selain itu, sistem perlu mendukung pembaruan data berdasarkan hasil analisis sentimen yang dipilih pengguna, sehingga informasi yang ditampilkan tetap relevan dengan dataset terbaru tanpa memerlukan pengambilan data ulang secara manual pada setiap komponen antarmuka.
  Selain kebutuhan fungsional, sistem juga memiliki kebutuhan non-fungsional yang berkaitan dengan kualitas pengelolaan data dan performa aplikasi \textit{frontend}. Kebutuhan non-fungsional tersebut meliputi konsistensi data antar-komponen antarmuka, efisiensi dalam pengambilan data dari API, serta mekanisme pengelolaan cache untuk mengurangi permintaan data yang tidak diperlukan. Sistem diharapkan mampu mengelola data secara terpusat di sisi \textit{frontend} sehingga setiap komponen menggunakan sumber data yang sama dan tersinkronisasi.
  Kebutuhan lain yang menjadi perhatian pada tahap ini adalah kebutuhan kemudahan pengembangan dan pemeliharaan sistem. Struktur pengelolaan data \textit{frontend} harus memungkinkan penambahan atau perubahan komponen dashboard tanpa memengaruhi keseluruhan sistem secara signifikan. Dengan demikian, spesifikasi kebutuhan ini menjadi dasar dalam perancangan arsitektur \textit{frontend} dan penerapan \textit{Client Data Layer} pada tahap desain dan implementasi selanjutnya.
\subsection{\textit{Design}}
  Tahap \textit{design} merupakan tahapan perancangan solusi berdasarkan spesifikasi kebutuhan sistem yang telah dirumuskan pada tahap \textit{requirement specification}. Pada tahap ini, kebutuhan sistem yang bersifat konseptual diterjemahkan ke dalam rancangan teknis yang menjadi acuan dalam proses implementasi \textit{frontend} dashboard analisis sentimen. Perancangan difokuskan pada bagaimana sistem \textit{frontend} mengelola dan menyajikan data secara terstruktur, konsisten, dan mudah dikembangkan sesuai dengan tujuan penelitian.
  Pada penelitian ini, tahap \textit{design} mencakup beberapa aspek utama, yaitu \textit{use case diagram}, \textit{entity relationship diagram}, alur sistem keseluruhan, perancangan arsitektur \textit{frontend}, perancangan \textit{Client Data Layer} sebagai mekanisme pengelolaan data dari API, serta perancangan antarmuka pengguna dalam bentuk \textit{wireframe}. Setiap aspek perancangan memiliki peran yang saling berkaitan dalam membangun sistem \textit{frontend} yang bersifat \textit{data-driven}.
  \begin{packed_enum}

    % \vspace{2cm}

    \item Use Case Diagram \hfill \\
    % \vspace{-.5cm}
    
      \begin{figure}[H]
        \centering
        \includegraphics[width=0.7\textwidth]{gambar/usecase-TA.png}
        \caption{Use Case Diagram}
        \label{fig:use-case}
      \end{figure}
      \FloatBarrier

      Use Case Diagram pada Gambar \ref{fig:use-case} menggambarkan interaksi utama antara aktor dan sistem Dashboard Analisis Sentimen yang dikembangkan dalam penelitian ini. Aktor utama pada sistem adalah User, yang merepresentasikan pengguna akhir seperti pelaku UMKM atau pemilik bisnis yang memanfaatkan sistem untuk memperoleh informasi analisis sentimen dari data media sosial. Interaksi pengguna dengan sistem diawali melalui use case Akses Landing Page, yang dapat diakses tanpa proses autentikasi. Dari use case ini, pengguna memiliki opsi untuk melakukan Register sebagai pengguna baru atau Login bagi pengguna yang telah terdaftar, yang dimodelkan menggunakan relasi extend karena kedua proses tersebut bersifat opsional dan bergantung pada kondisi pengguna.

      Setelah berhasil melakukan Login, pengguna dapat mengakses fitur-fitur utama sistem. Salah satu fitur utama adalah Melihat Sentiment, yang digunakan untuk menampilkan hasil analisis sentimen dari data yang telah diproses. Use case Melihat Sentiment memiliki relasi include dengan Menganalisa Data, yang menunjukkan bahwa proses analisis data merupakan bagian yang tidak terpisahkan dari penyajian informasi sentimen. Hasil analisis tersebut kemudian dapat diperluas melalui relasi extend ke use case Melihat Overall Sentiment, Insight Summary, dan Aspect Summary, yang memungkinkan pengguna memperoleh ringkasan sentimen secara keseluruhan, wawasan utama, serta analisis sentimen berbasis aspek.

      Selain itu, sistem juga menyediakan fitur Melihat Data Scrapper yang memungkinkan pengguna melihat data hasil proses pengambilan data dari media sosial. Dari use case ini, terdapat relasi extend ke Mengunduh Data, yang menunjukkan bahwa proses pengunduhan bersifat opsional dan hanya dilakukan apabila pengguna membutuhkan data tersebut dalam bentuk berkas. Di sisi lain, pengguna juga dapat mengakses fitur Melihat Rekomendasi Konten, yang disediakan sebagai bagian dari layanan sistem untuk memberikan rekomendasi berdasarkan hasil analisis sentimen. Keseluruhan relasi use case ini menggambarkan alur penggunaan sistem secara terstruktur, mulai dari akses awal hingga pemanfaatan fitur analisis dan penyajian informasi sentimen.
    \vspace*{3.3cm}
    \item Entity Relationship Diagram \hfill \\
     
      \begin{figure}[H]
        \centering
        \includegraphics[width=0.7\textwidth]{gambar/erd.png}
        \caption{Entity Relationship Diagram}
        \label{fig:ERD}
      \end{figure}
      \FloatBarrier

      
      ERD pada Gambar \ref{fig:ERD} menggambarkan bagaimana data hasil pengambilan data media sosial, hasil analisis sentimen, serta rekomendasi konten disimpan dan saling terhubung dalam basis data.

      Entitas users berperan sebagai entitas utama yang menyimpan informasi pengguna sistem, termasuk identitas dan data autentikasi. Setiap pengguna dapat menghasilkan satu atau lebih data hasil pengambilan media sosial yang direpresentasikan oleh entitas scrape\_results, yang menyimpan informasi profil serta data mentah hasil proses scraping. Relasi antara entitas users dan scrape\_results menunjukkan hubungan satu-ke-banyak, di mana satu pengguna dapat memiliki beberapa hasil pengambilan data.

      Hasil proses pengambilan data selanjutnya dianalisis dan disimpan pada entitas sentiment\_result, yang memiliki relasi satu-ke-satu atau satu-ke-banyak dengan scrape\_results tergantung pada kebutuhan analisis. Untuk mendukung analisis sentimen berbasis aspek, entitas sentiment\_comments digunakan untuk menyimpan komentar beserta kategori aspek sentimen seperti food\_quality, price, dan service. Entitas ini memiliki relasi satu-ke-banyak dengan sentiment\_result, yang memungkinkan satu hasil analisis sentimen memiliki banyak komentar terkait.

      Selain analisis sentimen, sistem juga menyediakan fitur rekomendasi konten yang dimodelkan melalui entitas recommendation\_result. Entitas ini berelasi dengan sentiment\_result sebagai dasar penyusunan rekomendasi. Detail rekomendasi disimpan pada beberapa entitas turunan, yaitu recommendation\_hashtags, recommendation\_captions, dan recommendation\_best\_posting, yang masing-masing memiliki relasi satu-ke-banyak terhadap recommendation\_result. Struktur ini memungkinkan sistem menyimpan berbagai bentuk rekomendasi secara terpisah namun tetap terhubung pada satu konteks hasil analisis.

      Selain itu, entitas langchain\_documents digunakan untuk menyimpan data teks, metadata, dan embedding vektor yang mendukung proses analisis lanjutan atau pemrosesan berbasis kecerdasan buatan. Langchain\_documents berfungsi sebagai penyimpanan dokumen pendukung di luar alur utama pengelolaan data dashboard.
      % \vspace{3.15cm}
    \item Alur Sistem Keseluruhan \hfill \\
  
    
      \begin{figure}[H]
        \centering
        \includegraphics[width=0.8\textwidth]{gambar/diagram-sistem-keseluruhan.png}
        \caption{Alur Sistem Keseluruhan}
        \label{fig:flow}
      \end{figure}
      \FloatBarrier

      Alur kerja keseluruhan sistem dimulai dari Dashboard, yang berfungsi sebagai pusat interaksi utama bagi pengguna. Dari dashboard, pengguna dapat mengakses halaman Scraper untuk melakukan proses pengambilan data media sosial. Pada tahap ini, pengguna memanfaatkan layanan scraper yang terintegrasi dengan extension scraper untuk mengumpulkan data dari media sosial. Setelah proses pengambilan data selesai, data hasil scraping dikirimkan ke backend sistem dan selanjutnya disimpan ke dalam \textit{database} sebagai data mentah.

      Data hasil scraping yang telah tersimpan kemudian dapat diproses lebih lanjut melalui tahap analisis data, di mana sistem melakukan analisis sentimen berbasis aspek (Aspect-Based Sentiment Analysis/ABSA). Proses ABSA ini menghasilkan informasi sentimen yang terstruktur berdasarkan aspek-aspek tertentu. Hasil analisis sentimen tersebut selanjutnya diteruskan ke modul rekomendasi konten, yang berfungsi untuk menghasilkan rekomendasi konten berdasarkan pola dan hasil sentimen yang diperoleh.

      Setelah proses rekomendasi selesai, data rekomendasi konten beserta hasil analisis sentimen dikirim kembali ke backend dan disimpan ke dalam \textit{database}. Pada tahap akhir, sistem menyajikan kembali data hasil scraping, hasil analisis sentimen, serta rekomendasi konten ke Dashboard, sehingga pengguna dapat melihat, mengevaluasi, dan memanfaatkan seluruh informasi yang dihasilkan oleh sistem secara terintegrasi. 
    
    \item Perancangan Arsitektur \textit{Frontend} \hfill \\
      Penelitian ini menerapkan prinsip \textit{component-based architecture}, di mana antarmuka pengguna dibangun dari komponen-komponen modular yang memiliki tanggung jawab spesifik dan dapat digunakan kembali. Setiap komponen difokuskan pada penyajian data dan interaksi pengguna, sementara logika pengelolaan data dipisahkan ke dalam lapisan tersendiri agar struktur aplikasi lebih terorganisasi dan mudah dipelihara.
      Arsitektur \textit{frontend} dirancang dengan pendekatan \textit{data-driven}, di mana tampilan antarmuka sepenuhnya bergantung pada data yang dikelola oleh sistem. Untuk mendukung hal tersebut, prinsip \textit{separation of concerns} diterapkan dengan memisahkan lapisan presentasi dan lapisan pengelolaan data, sehingga komponen antarmuka tidak berinteraksi langsung dengan REST API.
      \begin{figure}[H]
        \centering
        \includegraphics[width=0.5\textwidth]{diagram-architecture.png}
        \caption{Diagram Arsitektur \textit{Frontend}}
        \label{fig:frontend-architecture}
      \end{figure}
      \FloatBarrier

      Diagram arsitektur \textit{frontend} pada Gambar~\ref{fig:frontend-architecture} digunakan untuk menggambarkan pembagian lapisan sistem serta alur pengelolaan data pada sistem yang dikembangkan. Diagram ini menunjukkan bagaimana data dari REST API dikelola melalui \textit{Client Data Layer} sebelum disajikan pada komponen antarmuka pengguna.
      Lapisan REST API diposisikan sebagai sumber data eksternal yang menyediakan data hasil analisis sentimen. Seluruh proses pengolahan data, termasuk pengambilan data media sosial dan analisis sentimen, dilakukan pada sisi backend dan berada di luar ruang lingkup penelitian ini. \textit{Frontend} berperan sebagai konsumen data yang mengakses informasi tersebut melalui antarmuka REST API.
      Lapisan \textit{Client Data Layer} berfungsi sebagai lapisan perantara antara REST API dan komponen antarmuka pengguna. Data yang diperoleh dari REST API dikelola dan dimodelkan secara terpusat sebelum disajikan pada komponen antarmuka pengguna. Lapisan ini bertanggung jawab dalam proses pengambilan data, penyimpanan sementara (\textit{caching}), serta sinkronisasi data antar-komponen.
      Lapisan \textit{UI Components} merupakan lapisan presentasi yang bertugas menampilkan data kepada pengguna dan menangani interaksi pengguna dengan sistem. Komponen pada lapisan ini menerima data yang telah dikelola oleh \textit{Client Data Layer} dan menyajikannya dalam bentuk visualisasi seperti grafik dan tabel.
    \item Perancangan \textit{Client Data Layer} \hfill \\
      Perancangan \textit{Client Data Layer} dilakukan untuk mengelola data yang bersumber dari backend secara terpusat pada sisi \textit{frontend} sebelum disajikan pada komponen antarmuka pengguna. Pada aplikasi dashboard yang bersifat data-driven, data yang sama dapat digunakan oleh berbagai komponen secara bersamaan dan diperbarui secara dinamis. Oleh karena itu, diperlukan suatu lapisan pengelolaan data yang mampu mengatur alur data, menjaga konsistensi informasi, serta mengendalikan interaksi antara \textit{frontend} dan REST API.
      \textit{Client Data Layer} diposisikan sebagai lapisan perantara antara REST API dan komponen antarmuka pengguna, sebagaimana ditunjukkan pada Gambar \ref{fig:frontend-architecture} diagram arsitektur \textit{frontend}. Seluruh data yang diperoleh dari backend tidak langsung digunakan oleh komponen antarmuka, melainkan terlebih dahulu dikelola melalui \textit{Client Data Layer}. Dengan pendekatan ini, komponen antarmuka tidak perlu mengetahui detail proses pengambilan data dari API, sehingga fokus komponen dapat diarahkan pada penyajian data dan interaksi pengguna.
      Secara konseptual, \textit{Client Data Layer} memiliki beberapa tanggung jawab utama dalam sistem \textit{frontend}. Tanggung jawab tersebut meliputi proses pengambilan data dari REST API, pengelolaan server state, penyimpanan sementara data melalui mekanisme caching, serta sinkronisasi data antar-komponen antarmuka. Dengan pengelolaan data yang terpusat, permintaan data yang bersifat berulang dapat dikendalikan dan setiap komponen antarmuka memperoleh data yang konsisten sesuai dengan kondisi sistem.
      Selain pengelolaan alur data, \textit{Client Data Layer} juga dirancang untuk menangani pemodelan data sebelum digunakan oleh komponen antarmuka. Data yang diperoleh dari REST API dimodelkan secara terstruktur pada \textit{Client Data Layer} agar memiliki bentuk dan konsistensi yang jelas. Pendekatan ini bertujuan untuk meminimalkan ketergantungan komponen antarmuka terhadap struktur data mentah dari backend serta mempermudah proses pengembangan dan pemeliharaan sistem \textit{frontend}.
      Dalam penelitian ini, \textit{Client Data Layer} dirancang untuk diimplementasikan menggunakan TanStack Query sebagai pustaka pengelolaan server state pada \textit{frontend}. Pemilihan TanStack Query didasarkan pada kemampuannya dalam menyediakan mekanisme pengelolaan data asinkron secara terpusat, termasuk caching, sinkronisasi data, dan pengendalian permintaan data. Dengan memanfaatkan pustaka tersebut, \textit{Client Data Layer} diharapkan mampu mendukung pengelolaan data \textit{frontend} yang lebih terstruktur, konsisten, dan efisien sesuai dengan kebutuhan dashboard analisis sentimen.
    \item Perancangan Penggunaan TanStack Query \hfill \\
      Perancangan penggunaan TanStack Query pada penelitian ini mengacu pada dokumentasi resmi TanStack Query sebagai pustaka server state management untuk aplikasi \textit{frontend}. Berdasarkan dokumentasi resmi TanStack Query \citep{TanstackLCC2025}, pustaka ini dirancang untuk mengelola data asinkron yang bersumber dari API secara terpusat melalui mekanisme pengambilan data, penyimpanan sementara (caching), serta sinkronisasi data antar-komponen antarmuka. Pendekatan tersebut memungkinkan komponen \textit{frontend} memperoleh data yang konsisten tanpa harus melakukan permintaan data secara langsung ke REST API, sehingga pemisahan tanggung jawab antara lapisan presentasi dan lapisan pengelolaan data dapat terjaga. Dengan karakteristik tersebut, TanStack Query dinilai sesuai untuk diimplementasikan sebagai \textit{Client Data Layer} pada aplikasi \textit{frontend} yang bersifat data-driven, khususnya dalam konteks dashboard analisis sentimen yang membutuhkan konsistensi data dan pembaruan informasi secara terkontrol.
      \begin{figure}[H]
        \centering
        \includegraphics[width=0.5\textwidth]{gambar/tanstack-works.jpg}
        \caption{Cara Kerja TanStack Query}
        \label{fig:tanstack-works}
      \end{figure}
      \FloatBarrier

      Gambar \ref{fig:tanstack-works} menunjukkan alur pengelolaan data asinkron pada sisi \textit{frontend} menggunakan TanStack Query. Ketika komponen antarmuka membutuhkan data, permintaan tidak langsung dikirimkan ke REST API, melainkan terlebih dahulu dikelola oleh \textit{Client Data Layer}.
      TanStack Query melakukan pengecekan terhadap cache untuk menentukan apakah data yang diminta masih valid. Jika data tersedia dan masih relevan, data langsung dikembalikan ke komponen antarmuka tanpa melakukan pemanggilan ulang ke server. Sebaliknya, apabila data tidak tersedia atau sudah tidak valid, sistem akan melakukan proses fetching ke REST API dan menyimpan hasilnya ke dalam cache sebelum disajikan ke antarmuka pengguna.
      Mekanisme ini memungkinkan pengelolaan data yang lebih efisien, mengurangi jumlah permintaan API yang tidak perlu, serta menjaga konsistensi data antar-komponen pada \textit{frontend} dashboard analisis sentimen.
    \item Perancangan Antarmuka \hfill \\
      Rancangan antarmuka dilakukan sebagai tindak lanjut dari struktur arsitektur yang telah ditetapkan, dengan tujuan memastikan bahwa data hasil analisis sentimen yang dikelola oleh sistem dapat disajikan kepada pengguna secara informatif, mudah dipahami, dan konsisten. Antarmuka pengguna dirancang untuk merepresentasikan kebutuhan fungsional sistem dalam bentuk visual, sekaligus menjadi media interaksi antara pengguna dan sistem \textit{frontend} dashboard analisis sentimen.
      Sebagai bentuk konkret dari perancangan antarmuka pengguna, dilakukan penyusunan wireframe yang menggambarkan tata letak komponen, alur navigasi, serta penyajian informasi pada setiap halaman utama sistem.
       \begin{figure}[H]
        \centering
        \includegraphics[width=0.7\textwidth]{gambar/landing-page.jpeg}
        \caption{Wireframe Landing Page}
        \label{fig:landing-page}
      \end{figure}
      
      Landing Page pada Gambar \ref{fig:landing-page} dirancang sebagai halaman awal yang pertama kali diakses oleh pengguna ketika membuka sistem. Halaman ini berfungsi untuk memberikan gambaran umum mengenai tujuan dan fitur utama sistem sebelum pengguna melakukan proses autentikasi. Struktur halaman dirancang sederhana dengan penekanan pada informasi pengenalan sistem serta elemen navigasi utama yang mengarahkan pengguna ke halaman login atau registrasi. Perancangan wireframe ini bertujuan untuk memastikan pengguna dapat memahami konteks sistem secara cepat dan memiliki alur navigasi yang jelas menuju fitur utama yang disediakan.
      \begin{figure}[H]
        \centering
        \includegraphics[width=0.7\textwidth]{gambar/login.jpeg}
        \caption{Wireframe Halaman Login}
        \label{fig:login}
      \end{figure}
      Halaman Login pada Gambar \ref{fig:login} dirancang sebagai antarmuka autentikasi pengguna untuk mengakses fitur utama sistem. Halaman ini menyediakan elemen input untuk memasukkan kredensial pengguna untuk memproses autentikasi. pengguna dapat melakukan proses login sebelum diarahkan ke halaman dashboard. Halaman ini dirancang untuk mendukung keamanan akses sistem dengan memastikan bahwa hanya pengguna yang telah terautentikasi yang dapat mengakses fitur-fitur utama aplikasi.
      \begin{figure}[H]
        \centering
        \includegraphics[width=0.7\textwidth]{gambar/register.jpeg}
        \caption{Wireframe Halaman Register}
        \label{fig:register}
      \end{figure}
      Halaman Register pada Gambar \ref{fig:register} dirancang sebagai antarmuka pendaftaran pengguna baru sebelum dapat mengakses sistem. Halaman ini menyediakan form untuk pengisian data pengguna yang diperlukan dalam proses registrasi. Pengguna dapat melakukan proses pendaftaran secara sistematis. Halaman ini juga berfungsi sebagai bagian dari mekanisme kontrol akses dengan memastikan bahwa data pengguna dikumpulkan dan diproses sebelum akun dapat digunakan untuk mengakses fitur utama sistem.
      \begin{figure}[H]
        \centering
        \includegraphics[width=0.7\textwidth]{gambar/dashboard.jpeg}
        \caption{Wireframe Halaman Dashboard}
        \label{fig:dashboard}
      \end{figure}
      Halaman Dashboard pada Gambar \ref{fig:dashboard} dirancang sebagai halaman utama setelah pengguna berhasil melakukan autentikasi. Halaman ini berfungsi sebagai pusat informasi yang menampilkan ringkasan data dan visualisasi utama dari sistem. Struktur dashboard dirancang untuk memudahkan pengguna dalam memantau kondisi data secara keseluruhan serta mengakses fitur-fitur utama melalui navigasi yang tersedia. Perancangan wireframe ini bertujuan untuk memastikan penyajian informasi yang terstruktur dan mudah dipahami, sehingga pengguna dapat memperoleh gambaran umum hasil analisis secara cepat sebelum melakukan eksplorasi data lebih lanjut.
      \begin{figure}[H]
        \centering
        \includegraphics[width=0.7\textwidth]{gambar/sentiment.png}
        \caption{Wireframe Halaman Sentiment}
        \label{fig:sentiment}
      \end{figure}
      Halaman Sentiment pada Gambar \ref{fig:sentiment} dirancang untuk menampilkan hasil analisis sentimen aspect based sentiment analysis secara lebih rinci dibandingkan halaman dashboard. Halaman ini menyajikan informasi sentimen dalam bentuk visualisasi data yang memudahkan pengguna dalam memahami distribusi dan kecenderungan sentimen. Perancangan struktur halaman difokuskan pada penyajian data yang terorganisasi dan mudah diinterpretasikan, sehingga pengguna dapat melakukan analisis sentimen secara lebih mendalam sesuai dengan kebutuhan informasi yang diinginkan.
      \begin{figure}[H]
        \centering
        \includegraphics[width=0.7\textwidth]{gambar/scraper.png}
        \caption{Wireframe Halaman Scraper}
        \label{fig:scraper}
      \end{figure}
      Halaman Scraper pada Gambar \ref{fig:scraper} dirancang sebagai antarmuka yang memfasilitasi proses pengumpulan dan pemantauan data yang bersumber dari media sosial. Halaman ini menyajikan data hasil proses scraping yang dilakukan pada sisi server, sehingga pengguna dapat mengetahui data yang tersedia dan data yang mana yang akan digunakan oleh sistem pada tahap analisis.
      \begin{figure}[H]
        \centering
        \includegraphics[width=0.7\textwidth]{gambar/recomendation.jpeg}
        \caption{Wireframe Halaman Recomendation}
        \label{fig:recomendation}
      \end{figure}
      Halaman Recommendation pada Gambar \ref{fig:recomendation} dirancang sebagai antarmuka yang menyajikan rekomendasi berdasarkan hasil analisis sentimen yang telah diproses secara otomatis oleh sistem. Halaman ini menampilkan informasi rekomendasi yang diperoleh dari pengolahan data sentimen, sehingga pengguna dapat memahami insight yang dihasilkan dari data media sosial. Hasil rekomendasi yang ditampilkan pada halaman ini juga disajikan secara lebih detail sebagai bagian dari informasi rekomendasi utama. Halaman ini bertujuan untuk memudahkan pengguna dalam mengakses rekomendasi dan memahami hasil analisis sentimen secara terstruktur.
      \begin{figure}[H]
        \centering
        \includegraphics[width=0.45\textwidth]{gambar/chatbot.jpeg}
        \caption{Wireframe Halaman Chatbot}
        \label{fig:chatbot}
      \end{figure}
      Halaman Chatbot pada Gambar \ref{fig:chatbot} dirancang sebagai antarmuka yang memfasilitasi interaksi tanya jawab antara pengguna dengan sistem. Halaman ini memungkinkan pengguna untuk mengajukan pertanyaan terkait data yang tersedia dan memperoleh jawaban yang relevan berdasarkan informasi yang telah diproses oleh sistem. Halaman ini bertujuan untuk memudahkan pengguna dalam mengakses informasi dan memperoleh insight yang dibutuhkan secara interaktif.
    \end{packed_enum}
\subsection{Coding (Implementation)}
  Tahap coding diawali dengan penyiapan lingkungan pengembangan \textit{frontend} dan penyesuaian struktur proyek agar selaras dengan arsitektur yang telah dirancang. Framework React dipilih sebagai fondasi antarmuka pengguna, sedangkan TanStack Query diterapkan sebagai mekanisme utama pengelolaan server state yang membentuk \textit{Client Data Layer}.

  Dalam mengimplementasikan struktur direktori dan manajemen pengambilan data (data \textit{fetching}), penelitian ini mengadopsi pendekatan arsitektur berlapis (\textit{layered architecture}) yang direkomendasikan dalam praktik produksi React modern \citep{Owusu2025}. Sesuai pola tersebut, implementasi dilakukan dengan memisahkan logika API ke dalam service layer, yang kemudian dibungkus menggunakan custom hooks berbasis TanStack Query. Pendekatan ini memastikan bahwa komponen UI hanya berinteraksi dengan hooks tanpa perlu mengetahui kompleksitas implementasi REST API secara langsung.

  Bersamaan dengan itu, dilakukan pula penyiapan komponen pendukung dan integrasi data dengan backend. Seluruh persiapan ini bertujuan untuk menjamin bahwa proses implementasi berjalan secara terstruktur, menjaga konsistensi kode, serta memenuhi spesifikasi kebutuhan sistem yang telah didefinisikan pada tahap sebelumnya.

\subsection{Testing}
  Metode pengujian yang digunakan dalam penelitian ini mengacu pada konsep black-box testing, dengan pendekatan scenario-based testing. Pengujian dilakukan dengan mengamati perilaku sistem berdasarkan skenario penggunaan tanpa memperhatikan struktur internal kode program. Pendekatan ini menempatkan sistem sebagai sebuah kesatuan yang diuji dari sudut pandang pengguna, sehingga pengujian difokuskan pada kesesuaian fungsi dan respons sistem terhadap alur penggunaan yang dirancang.
  Pemilihan pendekatan scenario-based testing didasarkan pada karakteristik sistem yang dikembangkan, yaitu dashboard analitik yang bersifat data-driven dan mengandalkan interaksi antar-komponen antarmuka. Oleh karena itu, pengujian diarahkan pada pengetesan perilaku sistem dalam menampilkan data, menjaga konsistensi informasi, serta merespons pembaruan data sesuai dengan kondisi yang terjadi.
  Ruang lingkup pengujian pada penelitian ini difokuskan pada perilaku sistem \textit{frontend}, khususnya pada pengelolaan data melalui \textit{Client Data Layer} dan penyajian data pada antarmuka pengguna. Pengujian tidak mencakup pengetesan terhadap algoritma analisis sentimen maupun proses pengolahan data pada sisi backend.
  Skenario pengujian disusun berdasarkan fitur utama sistem dan merepresentasikan alur penggunaan dari sudut pandang pengguna. Setiap skenario dirancang untuk mengevaluasi perilaku sistem \textit{frontend} dalam merespons interaksi pengguna serta memastikan kesesuaian fungsi sistem dengan rancangan yang telah ditetapkan. Skenario pengujian dalam penelitian ini terdiri atas beberapa pengujian sebagai berikut:
  \begin{packed_enum}
      \vspace*{2.5cm}
    \item Landing Page
    % \vspace{-0.5cm}
      \setlength{\arrayrulewidth}{0.8pt}
      \renewcommand{\arraystretch}{1.4}
      \setlength{\tabcolsep}{4pt}

      \begin{longtable}{|
      >{\centering\arraybackslash}p{1.5cm} |
      p{5cm} |
      p{2.75cm} |
      p{2cm} |
      p{3cm} |
      }
      \caption{Skenario Pengujian Landing Page}
      \label{tab:pengujian-landing-page} \\

      \hline
      \textbf{ID} &
      \textbf{Test Step} &
      \textbf{Skenario Pengujian} &
      \textbf{Data Uji} &
      \textbf{Expected Result} \\
      \hline
      \endfirsthead

      \hline
      \textbf{ID} &
      \textbf{Test Step} &
      \textbf{Skenario Pengujian} &
      \textbf{Data Uji} &
      \textbf{Expected Result} \\
      \hline
      \endhead

      \endfoot

      \hline
      \endlastfoot

      TC-LP-01 &
      \begin{enumerate}
        \item Pengguna membuka browser
        \item Pengguna mengakses URL aplikasi
        \item Sistem memuat halaman awal
      \end{enumerate}
      &
      Pengguna mengakses aplikasi tanpa melakukan autentikasi
      &
      --
      &
      Halaman landing ditampilkan dengan informasi sistem serta navigasi menuju halaman login dan registrasi. \\
      \hline

      \end{longtable}


    % \vspace{-0.5cm}
    \item Login
    % \vspace{-0.5cm}
      \setlength{\arrayrulewidth}{0.8pt}
      \renewcommand{\arraystretch}{1.4}
      \setlength{\tabcolsep}{4pt}

      \begin{longtable}{|
      >{\centering\arraybackslash}p{1.5cm} |
      p{5cm} |
      p{2.75cm} |
      p{2cm} |
      p{3cm} |
      }
      \caption{Skenario Pengujian Halaman Login}
      \label{tab:login-testing} \\

      \hline
      \textbf{ID} &
      \textbf{Test Step} &
      \textbf{Skenario Pengujian} &
      \textbf{Data Uji} &
      \textbf{Hasil yang Diharapkan} \\
      \hline
      \endfirsthead

      \hline
      \textbf{ID} &
      \textbf{Test Step} &
      \textbf{Skenario Pengujian} &
      \textbf{Data Uji} &
      \textbf{Hasil yang Diharapkan} \\
      \hline
      \endhead

      \endfoot

      \hline
      \endlastfoot

      TC-LG-01 &
      \begin{enumerate}
        \item Pengguna membuka halaman login
        \item Pengguna mengisi username dan kata sandi valid
        \item Pengguna menekan tombol login
      \end{enumerate}
      &
      Login dengan kredensial yang valid
      &
      username dan kata sandi valid
      &
      Sistem menerima kredensial dan mengarahkan pengguna ke halaman dashboard. \\
      \hline

      TC-LG-02 &
      \begin{enumerate}
        \item Pengguna membuka halaman login
        \item Pengguna mengisi username valid
        \item Pengguna mengisi kata sandi salah
        \item Pengguna menekan tombol login
      \end{enumerate}
      &
      Login dengan kata sandi yang salah
      &
      username valid, kata sandi salah
      &
      Sistem menampilkan pesan kesalahan dan tetap berada di halaman login. \\
      \hline

      TC-LG-03 &
      \begin{enumerate}
        \item Pengguna membuka halaman login
        \item Pengguna mengisi username yang tidak terdaftar
        \item Pengguna mengisi kata sandi
        \item Pengguna menekan tombol login
      \end{enumerate}
      &
      Login dengan username yang tidak terdaftar
      &
      username tidak terdaftar
      &
      Sistem menampilkan pesan bahwa akun tidak ditemukan. \\
      \hline

      TC-LG-04 &
      \begin{enumerate}
        \item Pengguna membuka halaman login
        \item Pengguna tidak mengisi username atau kata sandi
        \item Pengguna menekan tombol login
      \end{enumerate}
      &
      Login dengan field kosong
      &
      username atau kata sandi kosong
      &
      Sistem menampilkan validasi bahwa seluruh field wajib diisi. \\
      \hline

      TC-LG-05 &
      \begin{enumerate}
        \item Pengguna membuka halaman login
        \item Pengguna mengisi ulang kredensial yang benar
        \item Pengguna menekan tombol login
      \end{enumerate}
      &
      Login berhasil setelah percobaan gagal
      &
      Kredensial valid
      &
      Sistem menerima kredensial dan mengarahkan pengguna ke halaman dashboard. \\
      \hline

      \end{longtable}
    % \vspace{-0.5cm}
    \item Register
    %  \vspace{-0.5cm}
      \setlength{\arrayrulewidth}{0.8pt}
      \renewcommand{\arraystretch}{1.4}
      \setlength{\tabcolsep}{4pt}

      \begin{longtable}{|
      >{\centering\arraybackslash}p{1.5cm} |
      p{5cm} |
      p{2.75cm} |
      p{2cm} |
      p{3cm} |
      }
      \caption{Skenario Pengujian Halaman Register}
      \label{tab:register-testing} \\

      \hline
      \textbf{ID} &
      \textbf{Test Step} &
      \textbf{Skenario Pengujian} &
      \textbf{Data Uji} &
      \textbf{Hasil yang Diharapkan} \\
      \hline
      \endfirsthead

      \hline
      \textbf{ID} &
      \textbf{Test Step} &
      \textbf{Skenario Pengujian} &
      \textbf{Data Uji} &
      \textbf{Hasil yang Diharapkan} \\
      \hline
      \endhead

      \endfoot

      \hline
      \endlastfoot

      TC-RG-01 &
      \begin{enumerate}
        \item Pengguna membuka halaman register
        \item Pengguna mengisi seluruh field registrasi dengan data valid
        \item Pengguna menekan tombol daftar
      \end{enumerate}
      &
      Registrasi dengan data yang valid
      &
      Data registrasi lengkap dan valid
      &
      Sistem menyimpan data pengguna dan akun dapat digunakan untuk login. \\
      \hline

      TC-RG-02 &
      \begin{enumerate}
        \item Pengguna membuka halaman register
        \item Pengguna tidak mengisi salah satu atau beberapa field
        \item Pengguna menekan tombol daftar
      \end{enumerate}
      &
      Registrasi dengan data tidak lengkap
      &
      Salah satu atau beberapa field kosong
      &
      Sistem menampilkan pesan validasi bahwa seluruh data registrasi wajib diisi. \\
      \hline

      TC-RG-03 &
      \begin{enumerate}
        \item Pengguna membuka halaman register
        \item Pengguna mengisi username yang sudah terdaftar
        \item Pengguna menekan tombol daftar
      \end{enumerate}
      &
      Registrasi dengan username yang sudah terdaftar
      &
      username sudah terdaftar
      &
      Sistem menampilkan pesan bahwa akun sudah terdaftar. \\
      \hline

      TC-RG-04 &
      \begin{enumerate}
        \item Pengguna membuka halaman register
        \item Pengguna mengisi ulang data registrasi dengan benar
        \item Pengguna menekan tombol daftar
      \end{enumerate}
      &
      Registrasi berhasil setelah kesalahan sebelumnya
      &
      Data registrasi valid
      &
      Sistem menerima data dan memproses registrasi akun. \\
      \hline

      \end{longtable}
          % \vspace{-0.5cm}
    \item Dashboard
    % \vspace{-0.5cm}
      \setlength{\arrayrulewidth}{0.8pt}
      \renewcommand{\arraystretch}{1.4}
      \setlength{\tabcolsep}{4pt}

      \begin{longtable}{|
      >{\centering\arraybackslash}p{1.5cm} |
      p{5cm} |
      p{2.75cm} |
      p{2cm} |
      p{3cm} |
      }
      \caption{Skenario Pengujian Halaman Dashboard}
      \label{tab:dashboard-testing} \\

      \hline
      \textbf{ID} &
      \textbf{Test Step} &
      \textbf{Skenario Pengujian} &
      \textbf{Data Uji} &
      \textbf{Hasil yang Diharapkan} \\
      \hline
      \endfirsthead

      \hline
      \textbf{ID} &
      \textbf{Test Step} &
      \textbf{Skenario Pengujian} &
      \textbf{Data Uji} &
      \textbf{Hasil yang Diharapkan} \\
      \hline
      \endhead

      \endfoot

      \hline
      \endlastfoot

      TC-DB-01 &
      \begin{enumerate}
        \item Pengguna berhasil login
        \item Sistem mengarahkan pengguna ke halaman dashboard
      \end{enumerate}
      &
      Membuka dashboard setelah login
      &
      Data dashboard tersedia
      &
      Sistem menampilkan ringkasan data dan visualisasi utama pada dashboard. \\
      \hline

      TC-DB-02 &
      \begin{enumerate}
        \item Pengguna login ke sistem
        \item Sistem memuat halaman dashboard
        \item Data dashboard belum tersedia
      \end{enumerate}
      &
      Membuka dashboard dengan data belum tersedia
      &
      Data belum tersedia
      &
      Sistem menampilkan indikator pemuatan atau pesan informasi kepada pengguna. \\
      \hline

      TC-DB-03 &
      \begin{enumerate}
        \item Pengguna membuka dashboard
        \item Pengguna berpindah ke menu lain
        \item Pengguna kembali ke halaman dashboard
      \end{enumerate}
      &
      Navigasi keluar dan kembali ke dashboard
      &
      --
      &
      Sistem menampilkan data dashboard secara konsisten tanpa kehilangan data. \\
      \hline

      TC-DB-04 &
      \begin{enumerate}
        \item Pengguna berada di halaman dashboard
        \item Sistem menerima pembaruan data dari server
      \end{enumerate}
      &
      Pembaruan data dashboard dari server
      &
      Data diperbarui
      &
      Sistem memperbarui tampilan dashboard sesuai dengan data terbaru. \\
      \hline

      TC-DB-05 &
      \begin{enumerate}
        \item Pengguna membuka halaman dashboard
        \item Beberapa komponen memuat data dari sumber yang sama
      \end{enumerate}
      &
      Konsistensi data antar komponen dashboard
      &
      Data yang sama digunakan
      &
      Sistem menampilkan data yang konsisten pada seluruh komponen dashboard. \\
      \hline

      \end{longtable}
        % \vspace{4cm}
    \item Sentiment
        % \vspace{-0.5cm}
      \setlength{\arrayrulewidth}{0.8pt}
    \renewcommand{\arraystretch}{1.4}
    \setlength{\tabcolsep}{4pt}

    \begin{longtable}{|
    >{\centering\arraybackslash}p{1.5cm} |
      p{5cm} |
      p{2.75cm} |
      p{2cm} |
      p{3cm} |
      }
    \caption{Skenario Pengujian Halaman Dashboard Sentiment}
    \label{tab:sentiment-testing} \\

    \hline
    \textbf{ID} &
    \textbf{Test Step} &
    \textbf{Skenario Pengujian} &
    \textbf{Data Uji} &
    \textbf{Hasil yang Diharapkan} \\
    \hline
    \endfirsthead

    \hline
    \textbf{ID} &
    \textbf{Test Step} &
    \textbf{Skenario Pengujian} &
    \textbf{Data Uji} &
    \textbf{Hasil yang Diharapkan} \\
    \hline
    \endhead

    \endfoot

    \hline
    \endlastfoot

    TC-ST-01 &
    \begin{enumerate}
      \item Pengguna berhasil login
      \item Pengguna membuka halaman dashboard sentiment
    \end{enumerate}
    &
    Membuka halaman dashboard sentiment
    &
    Data sentimen tersedia
    &
    Sistem menampilkan visualisasi dan ringkasan hasil analisis sentimen. \\
    \hline

    TC-ST-02 &
    \begin{enumerate}
      \item Pengguna membuka halaman dashboard sentiment
      \item Data sentimen belum tersedia
    \end{enumerate}
    &
    Membuka dashboard sentiment dengan data belum tersedia
    &
    Data sentimen belum tersedia
    &
    Sistem menampilkan indikator pemuatan atau pesan informasi kepada pengguna. \\
    \hline

    TC-ST-03 &
    \begin{enumerate}
      \item Pengguna berada di halaman dashboard sentiment
      \item Sistem menerima pembaruan data sentimen dari server
    \end{enumerate}
    &
    Pembaruan data sentimen dari server
    &
    Data sentimen diperbarui
    &
    Sistem memperbarui visualisasi sesuai dengan data sentimen terbaru. \\
    \hline

    TC-ST-04 &
    \begin{enumerate}
      \item Pengguna membuka dashboard sentiment
      \item Pengguna berpindah ke halaman lain
      \item Pengguna kembali ke dashboard sentiment
    \end{enumerate}
    &
    Navigasi keluar dan kembali ke dashboard sentiment
    &
    --
    &
    Sistem menampilkan data sentimen secara konsisten tanpa kehilangan data. \\
    \hline

    TC-ST-05 &
    \begin{enumerate}
      \item Pengguna membuka dashboard sentiment
      \item Beberapa komponen memuat data sentimen yang sama
    \end{enumerate}
    &
    Konsistensi data sentimen antar komponen
    &
    Data sentimen yang sama digunakan
    &
    Sistem menampilkan data yang konsisten pada seluruh komponen visualisasi. \\
    \hline

    \end{longtable}
    % \vspace{-0.5cm}
    \item Recommendation
    % \vspace{-0.5cm}
      \setlength{\arrayrulewidth}{0.8pt}
      \renewcommand{\arraystretch}{1.4}
      \setlength{\tabcolsep}{4pt}
      \begin{longtable}{|
      >{\centering\arraybackslash}p{1.5cm} |
      p{5cm} |
      p{2.75cm} |
      p{2cm} |
      p{3cm} |
      }
      \caption{Skenario Pengujian Halaman Dashboard Rekomendasi Konten}
      \label{tab:recommendation-testing} \\

      \hline
      \textbf{ID} & \textbf{Fitur} & \textbf{Skenario Pengujian} & \textbf{Data Uji} & \textbf{Hasil yang Diharapkan} \\
      \hline
      \endfirsthead

      \hline
      \textbf{ID} & \textbf{Fitur} & \textbf{Skenario Pengujian} & \textbf{Data Uji} & \textbf{Hasil yang Diharapkan} \\
      \hline
      \endhead
      \endfoot

      \hline
      \endlastfoot

      TC-RC-01 & Dashboard Rekomendasi & Pengguna membuka halaman dashboard rekomendasi konten & Data rekomendasi tersedia & Sistem menampilkan daftar rekomendasi berdasarkan hasil analisis sentimen \\
      \hline
      TC-RC-02 & Dashboard Rekomendasi & Pengguna membuka halaman rekomendasi dengan data belum tersedia & Data rekomendasi belum tersedia & Sistem menampilkan indikator pemuatan atau pesan informasi \\
      \hline
      TC-RC-03 & Dashboard Rekomendasi & Pembaruan data rekomendasi dari server & Data rekomendasi diperbarui & Sistem memperbarui tampilan rekomendasi sesuai data terbaru \\
      \hline
      TC-RC-04 & Dashboard Rekomendasi & Navigasi ke halaman lain dan kembali ke dashboard rekomendasi & -- & Sistem menampilkan data rekomendasi secara konsisten \\
      \hline
      TC-RC-05 & Dashboard Rekomendasi & Konsistensi rekomendasi dengan data sentimen & Data sentimen terkait berubah & Rekomendasi konten menyesuaikan perubahan data sentimen \\
      \hline

      \end{longtable}
        % \vspace{-0.5cm}
    \item Scraper
        % \vspace{-0.5cm}
      \setlength{\arrayrulewidth}{0.8pt}
      \renewcommand{\arraystretch}{1.4}
      \setlength{\tabcolsep}{4pt}

      \begin{longtable}{|
      >{\centering\arraybackslash}p{1.5cm} |
      p{5cm} |
      p{2.75cm} |
      p{2cm} |
      p{3cm} |
      }
      \caption{Skenario Pengujian Halaman Data Scraper}
      \label{tab:scraper-testing} \\

      \hline
      \textbf{ID} &
      \textbf{Test Step} &
      \textbf{Skenario Pengujian} &
      \textbf{Data Uji} &
      \textbf{Hasil yang Diharapkan} \\
      \hline
      \endfirsthead

      \hline
      \textbf{ID} &
      \textbf{Test Step} &
      \textbf{Skenario Pengujian} &
      \textbf{Data Uji} &
      \textbf{Hasil yang Diharapkan} \\
      \hline
      \endhead

      \endfoot

      \hline
      \endlastfoot

      TC-SC-01 &
      \begin{enumerate}
        \item Pengguna membuka halaman data scraper
        \item Sistem memuat data hasil scraping
      \end{enumerate}
      &
      Membuka halaman data scraper
      &
      Data hasil scraping tersedia
      &
      Sistem menampilkan daftar data hasil scraping yang diperoleh dari server. \\
      \hline

      TC-SC-02 &
      \begin{enumerate}
        \item Pengguna membuka halaman data scraper
        \item Data hasil scraping belum tersedia
      \end{enumerate}
      &
      Membuka halaman scraper dengan data belum tersedia
      &
      Data belum tersedia
      &
      Sistem menampilkan indikator pemuatan atau pesan informasi kepada pengguna. \\
      \hline

      TC-SC-03 &
      \begin{enumerate}
        \item Pengguna membuka halaman data scraper
        \item Pengguna memilih salah satu dataset
      \end{enumerate}
      &
      Pemilihan data untuk dianalisis
      &
      Dataset tertentu dipilih
      &
      Sistem menandai data terpilih untuk digunakan pada proses analisis. \\
      \hline

      TC-SC-04 &
      \begin{enumerate}
        \item Pengguna membuka halaman data scraper
        \item Pengguna berpindah ke halaman lain
        \item Pengguna kembali ke halaman data scraper
      \end{enumerate}
      &
      Navigasi keluar dan kembali ke halaman data scraper
      &
      --
      &
      Sistem menampilkan data scraper secara konsisten tanpa kehilangan data. \\
      \hline

      \end{longtable}
        % \vspace{-0.5cm}
    \item Chatbot
        % \vspace{-0.5cm}
      \setlength{\arrayrulewidth}{0.8pt}
      \renewcommand{\arraystretch}{1.4}
      \setlength{\tabcolsep}{4pt}

     \begin{longtable}{|
      >{\centering\arraybackslash}p{1.5cm} |
      p{5cm} |
      p{2.75cm} |
      p{2cm} |
      p{3cm} |
      }
      \caption{Skenario Pengujian Chatbot}
      \label{tab:chatbot-testing} \\

      \hline
      \textbf{ID} &
      \textbf{Test Step} &
      \textbf{Skenario Pengujian} &
      \textbf{Data Uji} &
      \textbf{Hasil yang Diharapkan} \\
      \hline
      \endfirsthead

      \hline
      \textbf{ID} &
      \textbf{Test Step} &
      \textbf{Skenario Pengujian} &
      \textbf{Data Uji} &
      \textbf{Hasil yang Diharapkan} \\
      \hline
      \endhead

      \endfoot

      \hline
      \endlastfoot

      TC-CB-01 &
      \begin{enumerate}
        \item Pengguna berhasil login
        \item Pengguna berada di halaman dashboard
        \item Pengguna membuka panel chatbot yang bersifat \textit{sticky}
      \end{enumerate}
      &
      Membuka antarmuka chatbot pada dashboard
      &
      --
      &
      Sistem menampilkan antarmuka chatbot dan siap menerima input pengguna. \\
      \hline

      TC-CB-02 &
      \begin{enumerate}
        \item Pengguna membuka chatbot pada dashboard
        \item Pengguna memasukkan pertanyaan teks
        \item Pengguna mengirimkan pertanyaan
      \end{enumerate}
      &
      Mengirimkan pertanyaan melalui chatbot
      &
      Pertanyaan teks valid
      &
      Sistem menampilkan respons chatbot sesuai dengan pertanyaan yang diberikan. \\
      \hline

      TC-CB-03 &
      \begin{enumerate}
        \item Pengguna membuka chatbot
        \item Pengguna mengirimkan input kosong
      \end{enumerate}
      &
      Mengirimkan input kosong pada chatbot
      &
      Input kosong
      &
      Sistem menampilkan pesan validasi atau informasi bahwa input tidak valid. \\
      \hline

      TC-CB-04 &
      \begin{enumerate}
        \item Pengguna membuka chatbot
        \item Pengguna mengirimkan pertanyaan di luar konteks sistem
      \end{enumerate}
      &
      Mengirimkan pertanyaan di luar konteks sistem
      &
      Pertanyaan tidak relevan
      &
      Sistem menampilkan respons atau pesan informasi yang sesuai. \\
      \hline

      TC-CB-05 &
      \begin{enumerate}
        \item Pengguna membuka chatbot
        \item Pengguna melakukan beberapa interaksi berturut-turut
      \end{enumerate}
      &
      Interaksi berulang dengan chatbot
      &
      Beberapa pertanyaan berurutan
      &
      Sistem memproses interaksi chatbot secara berurutan dengan menonaktifkan input selama proses berlangsung dan mengaktifkannya kembali setelah respons ditampilkan.. \\
      \hline

      \end{longtable}
      \vspace*{1cm}
    \item Logout
      % \vspace{-0.5cm}
      \setlength{\arrayrulewidth}{0.8pt}
      \renewcommand{\arraystretch}{1.4}
      \setlength{\tabcolsep}{4pt}

      \begin{longtable}{|
      >{\centering\arraybackslash}p{1.5cm} |
      p{5cm} |
      p{2.75cm} |
      p{2cm} |
      p{3cm} |
      }
      \caption{Skenario Pengujian Logout}
      \label{tab:logout-testing} \\

      \hline
      \textbf{ID} &
      \textbf{Test Step} &
      \textbf{Skenario Pengujian} &
      \textbf{Data Uji} &
      \textbf{Hasil yang Diharapkan} \\
      \hline
      \endfirsthead

      \hline
      \textbf{ID} &
      \textbf{Test Step} &
      \textbf{Skenario Pengujian} &
      \textbf{Data Uji} &
      \textbf{Hasil yang Diharapkan} \\
      \hline
      \endhead

      \endfoot

      \hline
      \endlastfoot

      TC-LO-01 &
      \begin{enumerate}
        \item Pengguna berada dalam kondisi login
        \item Pengguna menekan tombol logout
      \end{enumerate}
      &
      Melakukan logout dari sistem
      &
      --
      &
      Sistem mengakhiri sesi pengguna dan mengarahkan ke halaman login. \\
      \hline

      TC-LO-02 &
      \begin{enumerate}
        \item Pengguna telah melakukan logout
        \item Pengguna mencoba mengakses halaman dashboard
      \end{enumerate}
      &
      Mengakses dashboard setelah logout
      &
      --
      &
      Sistem menolak akses dan mengarahkan pengguna ke halaman login. \\
      \hline

      TC-LO-03 &
      \begin{enumerate}
        \item Pengguna menutup sesi aplikasi
        \item Pengguna membuka kembali aplikasi
      \end{enumerate}
      &
      Membuka ulang aplikasi setelah logout
      &
      --
      &
      Sistem meminta pengguna untuk melakukan login kembali. \\
      \hline

      \end{longtable}

  \end{packed_enum}
% \subsection{\textit{Operation}}
%   Tahap \textit{operation} merupakan tahapan di mana sistem frontend dashboard analisis sentimen dijalankan dan digunakan dalam lingkungan operasional terbatas untuk memastikan seluruh fungsi berjalan sesuai dengan tujuan perancangan. Pada tahap ini, sistem digunakan untuk menampilkan data hasil analisis sentimen yang diperoleh dari backend, sehingga dapat diamati kestabilan aplikasi, konsistensi penyajian data, serta respons antarmuka pengguna.
%   Tahap operation bertujuan untuk memastikan bahwa aplikasi frontend yang telah dikembangkan dapat digunakan secara fungsional sebagai dashboard analitik, serta mendukung aktivitas pemantauan informasi sentimen oleh pengguna tanpa kendala utama.Tahapan ini dapat menggunakan layanan shared hosting maupun \textit{Virtual Private Server (VPS)} yang telah dikonfigurasi untuk menjalankan dan mempublikasikan website. 
% \subsection{\textit{Maintenance}}
%   Tahap \textit{maintenance} dilakukan untuk menjaga kinerja dan stabilitas sistem setelah digunakan pada tahap operation. Aktivitas pemeliharaan pada tahap ini mencakup perbaikan kesalahan minor yang ditemukan selama penggunaan sistem, penyesuaian teknis terhadap perubahan kebutuhan atau data, serta penyempurnaan mekanisme pengelolaan data frontend agar tetap berjalan secara optimal.
%   Tahap maintenance juga memungkinkan dilakukannya penyesuaian kecil pada struktur komponen atau mekanisme sinkronisasi data tanpa mengubah arsitektur utama sistem, sehingga sistem tetap selaras dengan tujuan penelitian dan dapat digunakan secara berkelanjutan selama periode penelitian.
% \subsection{\textit{Evolution}}
%   Tahap \textit{evolution} dalam metode Fountain merupakan fase pengembangan berkelanjutan yang berfokus pada penyempurnaan sistem, penambahan fitur baru, dan adaptasi berdasarkan umpan balik pengguna serta perubahan kebutuhan bisnis. Pada penelitian ini, tahap ini melibatkan proses iteratif untuk meningkatkan \textit{dashboard sentiment analysis}, dengan evaluasi kinerja arsitektur Client Data Layer berbasis Tanstack Query.

%   Pendekatan ini digunakan untuk observasi terhadap integrasi frontend-backend, konsistensi data sentimen dari \textit{scraping} hingga visualisasi. Hasil observasi menjadi dasar arahan pengembangan selanjutnya, seperti penambahan analisis sentimen untuk platform baru (TikTok, Instagram), peningkatan algoritma rekomendasi berbasis \textit{machine learning}, atau optimasi performa \textit{data scraper}.

%   Tahap ini juga mempertimbangkan umpan balik pengguna akhir dan dinamika bisnis UMKM, sehingga sistem dapat terus berevolusi. Contoh pengembangan pada iterasi berikutnya meliputi integrasi \textit{chatbot}, \textit{dashboard analytics} dengan visualisasi interaktif, serta modul prediktif tren pasar. Dengan demikian, sistem tidak hanya dievaluasi, tetapi secara aktif dikembangkan menjadi lebih matang, skalabel, dan relevan.