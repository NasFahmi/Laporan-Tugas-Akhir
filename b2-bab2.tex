%==================================================================
% Ini adalah bab 2
% Silahkan edit sesuai kebutuhan, baik menambah atau mengurangi \section, \subsection
%==================================================================

\chapter[TINJAUAN PUSTAKA]{\\ TINJAUAN PUSTAKA}

\section{Landasan Teori}

\subsection{UMKM dan Digitalisasi}

Usaha Mikro, Kecil, dan Menengah (UMKM) merupakan sektor usaha produktif yang memiliki peran penting dalam perekonomian, khususnya dalam menciptakan lapangan kerja dan mendorong pertumbuhan ekonomi lokal. UMKM umumnya memiliki karakteristik berupa skala usaha yang relatif kecil, keterbatasan modal, serta pengelolaan usaha yang masih sederhana. Dalam beberapa tahun terakhir, perkembangan teknologi digital telah mendorong UMKM untuk beradaptasi dengan perubahan lingkungan bisnis agar tetap mampu bersaing di tengah dinamika pasar yang semakin cepat.

Meskipun digitalisasi menawarkan berbagai peluang, UMKM masih menghadapi sejumlah tantangan dalam proses adopsinya. Tantangan tersebut meliputi rendahnya literasi digital, keterbatasan pemahaman dalam pemanfaatan teknologi informasi, serta kurangnya kemampuan dalam mengelola dan menganalisis data bisnis. Selain itu, salah satu kendala utama yang dihadapi UMKM adalah keterbatasan akses terhadap sumber daya yang dibutuhkan untuk mengembangkan usaha, seperti modal, informasi, dan teknologi \citep{Alviani2025}.

Permasalahan digitalisasi UMKM juga diperkuat oleh faktor demografis dan geografis. Menurut \citep{Sofyan2025}, rendahnya literasi digital, khususnya pada pelaku UMKM usia lanjut dan yang berada di wilayah terpencil, menjadi hambatan dalam proses transformasi digital. Selain itu, tidak semua UMKM memiliki perangkat pendukung dan akses jaringan internet yang memadai, sehingga proses digitalisasi belum dapat diterapkan secara merata.

Dalam konteks pemasaran, pemanfaatan teknologi digital, khususnya media sosial, telah menjadi salah satu sarana utama bagi UMKM untuk mempromosikan produk dan menjangkau konsumen secara lebih luas. Aktivitas pemasaran digital tersebut menghasilkan data interaksi konsumen dalam jumlah besar yang berpotensi memberikan informasi berharga mengenai persepsi dan preferensi pasar. Oleh karena itu, UMKM membutuhkan sistem informasi yang mampu mengolah dan menyajikan data tersebut secara terstruktur dan informatif. Keberadaan sistem informasi berbasis dashboard analitik menjadi penting untuk mendukung pengambilan keputusan berbasis data serta meningkatkan efektivitas strategi pemasaran UMKM di era digital \citep{Trulline2021}.


\subsection{Media Sosial sebagai Sumber Data}

Media sosial telah berkembang tidak hanya sebagai sarana komunikasi dan pemasaran, tetapi juga sebagai sumber data yang mencerminkan opini, persepsi, dan perilaku konsumen. Melalui media sosial, konsumen dapat mengekspresikan pengalaman serta keterlibatan mereka terhadap suatu merek melalui berbagai bentuk interaksi, seperti komentar, unggahan, tanda suka, dan aktivitas berbagi konten. Interaksi tersebut mencerminkan keterlibatan konsumen dan menghasilkan data yang bernilai untuk dianalisis lebih lanjut \citep{Mardhatilah2024}. Data yang dihasilkan dari media sosial umumnya bersifat tidak terstruktur dan terus bertambah secara dinamis, sehingga memerlukan sistem yang mampu mengelola dan menyajikan informasi tersebut secara terstruktur agar dapat dimanfaatkan secara optimal.

Data yang dihasilkan dari media sosial memiliki karakteristik bersifat tidak terstruktur, berjumlah besar, dan terus bertambah secara dinamis. Komentar dan ulasan konsumen umumnya berbentuk teks bebas yang mengandung opini subjektif, emosi, serta penilaian terhadap suatu produk atau layanan. Kondisi ini menyebabkan data media sosial sulit dianalisis secara manual. Oleh karena itu, diperlukan pendekatan sistematis untuk mengolah dan mengekstraksi informasi penting dari data tersebut agar dapat dimanfaatkan secara optimal.

Dalam konteks UMKM, data media sosial berpotensi memberikan wawasan penting terkait preferensi konsumen, tingkat kepuasan pelanggan, serta isu-isu yang sering muncul dalam interaksi publik. Informasi ini dapat dimanfaatkan sebagai dasar evaluasi strategi pemasaran dan pengambilan keputusan bisnis. Namun, tanpa dukungan sistem yang mampu mengelola dan menyajikan data secara terstruktur, potensi data media sosial tersebut sulit dimanfaatkan secara efektif.

Oleh karena itu, media sosial diposisikan sebagai salah satu sumber data utama dalam pengembangan sistem analitik bagi UMKM. Data yang diperoleh dari media sosial selanjutnya dapat diolah dan disajikan dalam bentuk informasi yang lebih ringkas dan mudah dipahami melalui dashboard analitik. Pendekatan ini memungkinkan UMKM untuk memantau persepsi konsumen dan dapat digunakan sebagai alat pengambilan keputusan.

\subsection{Analisis Sentimen pada Media Sosial}

Analisis sentimen merupakan pendekatan analitik yang digunakan untuk mengidentifikasi dan mengklasifikasikan opini atau sikap pengguna terhadap suatu objek, seperti produk, layanan, atau merek, berdasarkan data teks. Dalam konteks media sosial, analisis sentimen memanfaatkan komentar, ulasan, dan berbagai bentuk interaksi pengguna untuk menentukan kecenderungan sentimen yang umumnya dikategorikan ke dalam sentimen positif, negatif, atau netral \citep{Suryani2024}.

Seiring dengan meningkatnya aktivitas pengguna di media sosial, volume data opini yang dihasilkan semakin besar, bersifat dinamis, dan umumnya tidak terstruktur. Data tersebut mengandung unsur subjektivitas, emosi, serta bahasa informal, sehingga sulit dianalisis secara manual. Oleh karena itu, analisis sentimen digunakan untuk menyederhanakan data teks yang kompleks menjadi informasi yang lebih terstruktur agar dapat dimanfaatkan dalam memahami kecenderungan opini publik dan perilaku konsumen.

Penelitian oleh \citep{Fajarini2025} menunjukkan bahwa analisis sentimen berbasis data media sosial mampu memberikan wawasan mengenai pola opini publik, preferensi konsumen, serta perubahan tren yang relevan dalam konteks bisnis. Pendekatan ini memungkinkan pemantauan opini dan dinilai lebih efisien dibandingkan metode konvensional, seperti survei manual atau riset pasar tradisional.

Selain itu, penelitian lain yang dilakukan oleh \citep{Maharani2024} menegaskan bahwa pengolahan data media sosial melalui teknik data mining berperan penting dalam mengidentifikasi pola dan tren penggunaan data yang mencerminkan perilaku serta preferensi pengguna. Dalam konteks tersebut, analisis sentimen menjadi salah satu komponen yang berkontribusi dalam menggali makna dari data teks yang dihasilkan pengguna media sosial, sehingga wawasan yang diperoleh dapat dimanfaatkan untuk mendukung proses analisis dan pengambilan keputusan berbasis data.

Meskipun demikian, analisis sentimen juga memiliki keterbatasan, terutama terkait kualitas data yang bervariasi dan adanya noise dalam data media sosial yang dapat memengaruhi hasil interpretasi. Oleh karena itu, diperlukan pemahaman konteks dan pendekatan analisis yang tepat agar kesimpulan yang dihasilkan tidak bersifat keliru atau bias \citep{Suryani2024}.

Dalam penelitian ini, analisis sentimen diposisikan sebagai proses pengolahan data pada sisi backend. Fokus penelitian tidak terletak pada metode atau algoritma analisis sentimen yang digunakan, melainkan pada pemanfaatan hasil analisis sentimen sebagai sumber data yang dikelola dan disajikan melalui dashboard analitik di sisi \textit{frontend}.


\subsection{Dashboard Analitik dan Visualisasi Data}

Dashboard analitik merupakan sistem informasi yang dirancang untuk menyajikan data dalam bentuk visual yang ringkas, terintegrasi, dan mudah dipahami oleh pengguna. Dashboard ini umumnya memanfaatkan berbagai komponen visualisasi, seperti grafik, tabel, dan indikator kinerja, untuk menampilkan informasi penting yang mendukung proses pemantauan, analisis, dan pengambilan keputusan. Berbeda dengan laporan statis, dashboard analitik bersifat dinamis dan interaktif, sehingga memungkinkan pengguna untuk memperoleh gambaran kondisi secara cepat dan menyeluruh.

Visualisasi data memiliki peran penting dalam dashboard analitik karena mampu mengubah data mentah menjadi informasi yang lebih bermakna dan intuitif. Melalui visualisasi yang tepat, pengguna dapat dengan mudah mengidentifikasi pola, tren, serta perubahan yang terjadi pada data. Visualisasi data juga berfungsi sebagai sarana penyampaian informasi yang bersifat naratif, di mana kinerja dan kondisi bisnis dapat digambarkan secara komprehensif tanpa harus melalui proses analisis data yang kompleks.

Penelitian yang dilakukan oleh \citep{Rathore2025} menunjukkan bahwa penerapan teknik visualisasi data yang efektif dapat membantu pengambil keputusan dalam menghasilkan keputusan yang lebih informatif, akurat, dan ringkas. Studi tersebut menegaskan bahwa dashboard, berbagai jenis grafik, serta pendekatan visualisasi yang terstruktur berperan penting dalam meningkatkan kualitas \textit{business intelligence} dan mendukung perencanaan strategis. Selain itu, visualisasi data dinilai mampu meningkatkan efisiensi proses pengambilan keputusan dan mengurangi waktu yang dibutuhkan untuk menganalisis data, sehingga organisasi dapat merespons permasalahan dan dinamika bisnis secara lebih cepat.

Pentingnya analisis dan visualisasi data dalam konteks bisnis juga dikemukakan oleh \citep{Purnama2025}, yang menyatakan bahwa visualisasi data berperan dalam membantu organisasi memahami tren dan pola yang terdapat dalam data. Informasi yang dihasilkan dari proses analisis tersebut dapat dimanfaatkan untuk mendukung pengambilan keputusan serta perencanaan bisnis berbasis data.

Dalam konteks aplikasi data-driven, dashboard analitik berfungsi sebagai jembatan antara data dan pengambilan keputusan. Dashboard memungkinkan penyajian data yang informatif sehingga perubahan kondisi dan tren dapat dipantau secara berkelanjutan. Hal ini menjadikan dashboard analitik sebagai komponen penting dalam sistem yang memanfaatkan data berskala besar dan bersifat dinamis, termasuk data yang dihasilkan dari media sosial.

Bagi UMKM, keberadaan dashboard analitik menjadi sangat relevan karena dapat menyederhanakan informasi yang kompleks menjadi tampilan visual yang mudah dipahami. Informasi seperti kecenderungan sentimen konsumen, pola interaksi pengguna, dan ringkasan data pemasaran dapat disajikan secara visual, sehingga pelaku UMKM tidak perlu melakukan analisis data secara manual. Dengan demikian, dashboard analitik dapat mendukung UMKM dalam mengambil keputusan bisnis yang lebih cepat, tepat, dan berbasis data.

\subsection{\textit{Blackbox Testing}}

\textit{Blackbox testing} merupakan salah satu metode pengujian perangkat lunak yang berfokus pada pengujian fungsional sistem dari sudut pandang pengguna tanpa memperhatikan struktur internal atau implementasi kode program. Pada metode ini, sistem diperlakukan sebagai sebuah “kotak hitam”, di mana pengujian dilakukan dengan memberikan input tertentu dan kemudian mengamati output atau respons yang dihasilkan untuk menilai kesesuaian fungsi sistem dengan spesifikasi yang telah ditetapkan. Pendekatan ini banyak digunakan dalam pengujian perangkat lunak karena mampu mengevaluasi perilaku sistem secara langsung berdasarkan kebutuhan pengguna.

Metode \textit{blackbox testing} menitikberatkan pada validasi fungsi utama sistem, seperti proses input dan output, alur penggunaan, serta respons sistem terhadap berbagai kondisi penggunaan. Dengan demikian, metode ini sangat sesuai untuk menguji sistem berbasis antarmuka pengguna, khususnya aplikasi web dan dashboard, yang keberhasilannya sangat bergantung pada kesesuaian fungsi dan kemudahan penggunaan. Pengujian dilakukan tanpa melibatkan analisis terhadap logika internal atau struktur kode, sehingga hasil pengujian lebih berorientasi pada pengalaman dan kebutuhan pengguna akhir.

Penelitian yang dilakukan oleh Maulida et al. \citep{Maulida2025} menunjukkan bahwa \textit{blackbox testing} efektif digunakan dalam pengujian sistem website pemesanan online karena mampu mengidentifikasi kesalahan fungsional pada tahap awal pengembangan. Dalam penelitian tersebut, pengujian dilakukan pada berbagai fitur utama sistem, seperti proses registrasi, login, pencarian produk, hingga konfirmasi pesanan, tanpa meninjau kode program yang digunakan. Hasil penelitian tersebut membuktikan bahwa penerapan \textit{blackbox testing} dapat membantu memastikan kualitas sistem dari aspek fungsionalitas sebelum sistem dirilis kepada pengguna.

Berdasarkan karakteristik tersebut, black-box testing dipandang sebagai metode pengujian yang relevan untuk digunakan dalam penelitian ini. Metode ini memungkinkan evaluasi terhadap perilaku sistem frontend secara menyeluruh berdasarkan skenario penggunaan, sehingga kesesuaian fungsi sistem dengan kebutuhan pengguna dapat dinilai secara sistematis.


\subsection{Arsitektur \textit{Client Data Layer}}

Pada aplikasi frontend modern yang bersifat data-driven, pengelolaan data yang bersumber dari Application Programming Interface (API) menjadi salah satu aspek penting dalam arsitektur sistem. Aplikasi seperti dashboard analitik umumnya menampilkan berbagai komponen antarmuka, seperti grafik, tabel, dan indikator, yang bergantung pada data yang sama dan diperbarui berdasarkan sumber data dari API. Jika pengambilan dan pengolahan data dilakukan secara langsung di setiap komponen antarmuka, maka dapat menimbulkan berbagai permasalahan, seperti permintaan API yang berulang, inkonsistensi data antar-komponen, serta peningkatan beban render yang berdampak pada penurunan performa aplikasi.

\textit{Client Data Layer} merupakan pendekatan arsitektural dalam pengembangan frontend modern yang bertujuan untuk mengelola data yang bersumber dari server secara terpusat di sisi klien. Pendekatan ini muncul sebagai respons terhadap meningkatnya kompleksitas aplikasi data-driven, dimana data bersifat asinkron, dinamis, dan digunakan oleh banyak komponen antarmuka secara bersamaan. Dengan adanya \textit{Client Data Layer}, proses \textit{fetching}, \textit{caching}, dan \textit{synchronize data} dapat dilakukan secara lebih terstruktur, sehingga membantu menjaga konsistensi data dan mendukung pengelolaan data \textit{frontend} secara lebih terstruktur.

\begin{figure}[H]
  \centering
  \includegraphics[width=0.8\textwidth]{gambar/client-data-layer-paradigm.png}
  \caption{Arsitektur \textit{Client Data Layer}}
  \label{fig:client-data-layer-paradigm}
\end{figure}
\FloatBarrier

\begin{packed_enum}
  \item \textit{Data-driven Frontend Architecture}
  
  \textit{Data-driven Frontend Architecture} merupakan pendekatan pengembangan di mana seluruh antarmuka pengguna didorong oleh data sebagai sumber kebenaran utama. Dalam pola ini, UI dihasilkan sebagai fungsi dari state yang ada—artinya, setiap perubahan pada data akan secara otomatis memicu pembaruan pada tampilan aplikasi. Arsitektur ini sangat berguna untuk aplikasi \textit{frontend} yang menampilkan informasi dinamis seperti dashboard analitik, papan monitoring. Keunggulan utamanya adalah konsistensi tampilan yang lebih terjamin, alur data yang mudah dilacak, dan pengembangan fitur baru yang lebih sistematis karena UI dan logika data terpisah dengan jelas.

  \item \textit{Server State Management}

  \textit{Server state management} mengacu pada pengelolaan data yang bersumber dari sistem eksternal, seperti API atau layanan backend, yang bersifat asinkron dan dapat berubah di luar kendali langsung aplikasi klien. Data ini memiliki karakteristik dinamis karena dipengaruhi oleh kondisi jaringan, waktu respons server, serta pembaruan data di sisi backend. Oleh karena itu, server state memerlukan mekanisme khusus untuk menangani proses pengambilan data, status pemuatan, penanganan kesalahan, serta pembaruan dan sinkronisasi data agar informasi yang digunakan oleh berbagai komponen antarmuka tetap konsisten dan akurat.

  \item \textit{Client-side Data Management}
  
  \textit{Client-side data management} berfokus pada pengelolaan data di sisi klien setelah data tersebut diperoleh dari \textit{server}, termasuk penyimpanan sementara, penggunaan ulang data, serta distribusi data ke berbagai komponen antarmuka. Pendekatan ini bertujuan untuk mengurangi ketergantungan terhadap permintaan data berulang ke server sekaligus menjaga konsistensi informasi yang ditampilkan pada antarmuka pengguna. Dengan pengelolaan data yang terstruktur di sisi klien, aplikasi frontend dapat menyajikan data secara lebih responsif dan stabil, khususnya pada aplikasi frontend yang bersifat data-driven seperti dashboard analitik, tanpa mencampurkan pengelolaan data server dengan state antarmuka pengguna.
 
  \item \textit{State Management for Asynchronous Data}
  
  \textit{State Management for Asynchronous Data} adalah pendekatan khusus untuk menangani data yang diperoleh melalui state, seperti panggilan API, operasi file, atau permintaan jaringan lainnya. Karena data tersebut tidak tersedia secara instan, sistem harus mampu mengelola berbagai state yang mungkin terjadi: mulai dari state \textit{idle}, state \textit{fetching} atau \textit{onloading}, state \textit{success}, hingga state \textit{error}. Tantangan utamanya adalah memastikan aplikasi tetap responsif dan memberikan umpan balik yang informatif kepada pengguna selama proses pengambilan data. Pendekatan ini juga mencakup strategi seperti pembatalan permintaan yang tidak diperlukan, pengulangan otomatis saat gagal, dan pembaruan data latar belakang untuk menjaga informasi tetap konsisten.

  \item \textit{Frontend Data Caching and Synchronization}
  
  \textit{Frontend Data Caching and Synchronization} adalah teknik untuk meningkatkan kinerja aplikasi dengan menyimpan salinan data dari server di memori klien. Dengan adanya \textit{cache}, aplikasi dapat menampilkan informasi secara cepat tanpa perlu melakukan permintaan berulang ke server. Namun, teknik ini juga menimbulkan tantangan, yaitu data yang disimpan dapat menjadi kedaluwarsa jika terjadi perubahan di sisi server. Oleh karena itu, diperlukan mekanisme sinkronisasi yang cerdas, seperti pembaruan di latar belakang (background refresh) dan penandaan \textit{cache} yang kedaluwarsa (cache invalidation). Dengan pendekatan ini, aplikasi dapat menampilkan data secara lebih konsisten sekaligus menjaga keakuratan informasi yang ditampilkan.

  Beberapa penelitian menunjukkan bahwa penerapan mekanisme \textit{caching} pada aplikasi \textit{frontend} dapat meningkatkan performa dan responsivitas sistem. \textit{Caching} memungkinkan data hasil pemanggilan API disimpan sementara di sisi klien sehingga dapat digunakan kembali tanpa melakukan permintaan ulang ke server. Pendekatan ini terbukti mampu mengurangi duplikasi permintaan API dan mempercepat waktu respons aplikasi. Pemanfaatan pustaka seperti TanStack Query dalam mengelola \textit{caching} dan prefetching data juga dinilai efektif dalam meningkatkan efisiensi pengelolaan data di sisi \textit{frontend} serta pengalaman pengguna secara keseluruhan \citep{Rahman2024}. Dalam konteks penelitian ini, peningkatan performa dipahami sebagai perubahan perilaku pengelolaan data \textit{frontend} yang diamati melalui mekanisme \textit{caching} dan pengendalian permintaan data, tanpa dilakukan pengukuran performa numerik.

\end{packed_enum}

Perlu dibedakan secara konseptual antara \textit{client state} dan \textit{server state} dalam pengembangan aplikasi frontend modern. Client state mengacu pada state yang sepenuhnya berada di sisi klien dan berkaitan langsung dengan antarmuka pengguna, seperti status tampilan, input formulir, filter, atau kondisi navigasi. State ini bersifat sinkron, berumur pendek, dan hanya relevan dalam konteks sesi interaksi pengguna. Sebaliknya, server state merepresentasikan data yang bersumber dari sistem eksternal seperti API atau layanan backend, bersifat asinkron, dapat berubah di luar kendali aplikasi klien, serta memerlukan penanganan khusus terkait proses pengambilan data, status pemuatan, penanganan kesalahan, dan sinkronisasi. Perbedaan karakteristik tersebut menjadikan pengelolaan server state tidak efektif apabila disamakan dengan client state, sehingga diperlukan pendekatan arsitektural tersendiri berupa Client Data Layer untuk mengelola data server secara terpusat, konsisten, dan terstruktur.


\subsection{\textit{REST API}}

\textit{Representational State Transfer Application Programming Interface (REST API)} merupakan gaya arsitektur layanan web yang digunakan untuk memungkinkan komunikasi antara klien dan server melalui protokol HTTP secara terstandarisasi. REST API bersifat stateless, di mana setiap permintaan dari klien harus membawa seluruh informasi yang dibutuhkan untuk diproses oleh server, sehingga tidak bergantung pada status permintaan sebelumnya. Data yang dipertukarkan umumnya disajikan dalam format JSON karena bersifat ringan dan mudah diproses oleh aplikasi frontend, menjadikan REST API banyak digunakan pada aplikasi web modern yang bersifat data-driven.

Dalam konteks aplikasi \textit{frontend analitik}, REST API berperan sebagai sumber utama data (server state) yang dikonsumsi oleh antarmuka pengguna. Data yang diperoleh melalui REST API bersifat asinkron dan dapat berubah sewaktu-waktu, sehingga memerlukan mekanisme pengelolaan data yang mampu menangani proses pengambilan, pembaruan, dan sinkronisasi data secara efisien. Penelitian terkini menunjukkan bahwa REST API memiliki keunggulan dalam penyajian data yang bersifat datar dan mudah di-cache, sehingga mampu meningkatkan efisiensi distribusi data serta memperbesar rasio cache hit pada lapisan jaringan. Pendekatan ini dinilai lebih optimal untuk kebutuhan aplikasi yang menampilkan data terstruktur secara berulang, seperti dashboard dan sistem pelaporan, dibandingkan pendekatan API lain yang lebih kompleks \citep{Islam2025}.

Lebih lanjut, penelitian tersebut menegaskan bahwa performa aplikasi web tidak ditentukan oleh satu teknologi tertentu, melainkan oleh keselarasan antara desain akses data, mekanisme \textit{caching}, serta pengelolaan state pada sisi klien. REST API yang dirancang dengan kontrak data yang jelas dan \textit{cache-aware} terbukti mendukung peningkatan performa dan skalabilitas aplikasi ketika dipadukan dengan lapisan pengelolaan data di \textit{frontend}. Oleh karena itu, dalam pengembangan aplikasi \textit{frontend} modern, REST API umumnya tidak diakses secara langsung oleh setiap komponen antarmuka, melainkan melalui lapisan pengelolaan data seperti \textit{Client Data Layer} agar data dapat dikelola secara terpusat, konsisten, dan efisien.

Dalam penelitian ini, REST API diposisikan sebagai penyedia data hasil analisis sentimen yang diproses di sisi backend. Data tersebut selanjutnya dikelola pada sisi frontend melalui arsitektur \textit{Client Data Layer} sebelum ditampilkan dalam bentuk visualisasi pada dashboard analitik. Dengan pemisahan peran ini, REST API berfungsi sebagai sumber data, sementara pengelolaan performa, \textit{caching}, dan sinkronisasi data dilakukan sepenuhnya di sisi \textit{frontend} untuk mendukung penyajian informasi yang responsif dan konsisten.


Penerapan \textit{Client Data Layer} memberikan sejumlah manfaat dalam pengembangan aplikasi \textit{frontend}. Salah satu manfaat utama adalah peningkatan konsistensi data, di mana beberapa komponen yang membutuhkan data yang sama dapat memperoleh informasi yang seragam tanpa harus melakukan permintaan data secara terpisah. Selain itu, \textit{Client Data Layer} memungkinkan pengurangan jumlah permintaan API yang tidak diperlukan melalui mekanisme caching dan pengelolaan siklus data. Pendekatan ini juga mendukung penyajian data yang lebih stabil dan terkelola serta mempermudah pengelolaan data yang bersifat asinkron dan dinamis.

Dalam konteks dashboard analitik, keberadaan \textit{Client Data Layer} menjadi semakin penting karena data yang ditampilkan umumnya bersifat besar, sering diperbarui, dan digunakan oleh banyak komponen secara bersamaan. Dengan memanfaatkan \textit{Client Data Layer}, dashboard dapat menampilkan data secara lebih responsif dan stabil, sekaligus meminimalkan risiko inkonsistensi informasi yang ditampilkan kepada pengguna. Pendekatan ini mendukung terciptanya arsitektur \textit{frontend} yang lebih terorganisasi, mudah dipelihara, dan skalabel.

\subsection{React}

React merupakan sebuah pustaka (\textit{library}) JavaScript yang digunakan untuk membangun antarmuka pengguna (user interface) pada aplikasi web yang bersifat interaktif dan responsif. Menurut dokumentasi resmi React, pustaka ini dirancang untuk membangun antarmuka dengan pendekatan \textit{component-based}, di mana tampilan antarmuka dibagi menjadi bagian-bagian kecil yang dapat digunakan kembali (\textit{reusable components}) sehingga memudahkan pengelolaan dan pemeliharaan kode aplikasi \citep{ReactOfficial}. React bekerja dengan memanfaatkan virtual DOM untuk meminimalkan operasi pada DOM aktual sehingga perubahan tampilan dapat dilakukan secara efisien saat data atau state aplikasi berubah, tanpa perlu melakukan refresh seluruh halaman.

Pendekatan \textit{component-based architecture} yang digunakan React juga memungkinkan pengembang merancang antarmuka sebagai susunan komponen modular yang saling terpisah namun saling berinteraksi, yang selaras dengan prinsip pengembangan frontend modern. Struktur seperti ini tidak hanya meningkatkan keterbacaan dan modularitas kode, tetapi juga memberikan fleksibilitas dalam pengembangan aplikasi berskala besar serta mempermudah pengujian dan pemeliharaan. Dalam konteks pengembangan aplikasi \textit{Single Page Application (SPA)} dan \textit{API-driven application}, studi yang dilakukan pada tren pengembangan aplikasi web menunjukkan bahwa penggunaan React dalam kombinasi dengan arsitektur API terbukti mendukung pengembangan aplikasi web yang modular, fleksibel, dan mudah dikembangkan secara berkelanjutan \citep{TrendsWebDevReact2025}.

Dengan karakteristik tersebut, React dipilih sebagai teknologi \textit{frontend} dalam penelitian ini untuk membangun antarmuka pengguna yang dinamis dan dapat berinteraksi secara langsung dengan lapisan pengelolaan data (\textit{Client Data Layer}), sehingga mendukung kebutuhan aplikasi yang bersifat \textit{data-driven}.

\subsection{TanStack Query (React Query)}

TanStack Query merupakan pustaka manajemen data pada sisi \textit{frontend} yang dirancang untuk mengelola data yang bersumber dari server (\textit{server state}) secara efisien. Dalam dokumentasi resminya, TanStack Query dijelaskan sebagai \textit{"the missing data-fetching layer for web applications"} yang berfungsi untuk mempermudah proses pengambilan, penyimpanan sementara (caching), sinkronisasi, serta pembaruan data dari server \citep{TanstackLCC2025}. Pendekatan ini ditujukan untuk menangani kompleksitas data asinkron yang tidak dapat dikelola secara optimal menggunakan mekanisme state management konvensional.

Dalam dokumentasi resminya, TanStack Query mendefinisikan server state sebagai data yang berasal dari sumber eksternal dan memiliki karakteristik asinkron, dapat berubah sewaktu-waktu, serta memerlukan mekanisme khusus untuk menjaga konsistensi data di sisi klien. Oleh karena itu, TanStack Query menyediakan pendekatan deklaratif dalam pengelolaan server state, di mana pengembang dapat mendefinisikan kebutuhan data tanpa harus menangani secara manual proses sinkronisasi dan pembaruan data di setiap komponen antarmuka (TanStack Documentation, 2024).

Salah satu fitur utama TanStack Query adalah mekanisme \textit{caching} yang memungkinkan data hasil pemanggilan API disimpan sementara di sisi klien. Dengan adanya \textit{caching}, data yang telah diperoleh dapat digunakan kembali oleh komponen lain tanpa perlu melakukan permintaan ulang ke server, selama data tersebut masih dianggap valid. Pendekatan ini berkontribusi dalam mengurangi jumlah permintaan API yang tidak diperlukan, meningkatkan efisiensi aplikasi, serta mempercepat waktu respons antarmuka pengguna.

Selain \textit{caching}, TanStack Query juga menyediakan mekanisme sinkronisasi data yang mendukung pembaruan data secara otomatis. Melalui konsep seperti refetching dan invalidasi data, TanStack Query memastikan bahwa data yang ditampilkan tetap mutakhir ketika terjadi perubahan di sisi server. Mekanisme ini sangat relevan pada aplikasi data-driven, seperti dashboard analitik, yang menampilkan data secara dinamis dan digunakan oleh banyak komponen secara bersamaan.

Dalam konteks arsitektur frontend, TanStack Query dapat diposisikan sebagai implementasi konkret dari \textit{Client Data Layer}. Pustaka ini berperan sebagai lapisan perantara antara backend API dan komponen antarmuka pengguna, sehingga komponen UI tidak berinteraksi langsung dengan API. Dengan demikian, TanStack Query membantu memisahkan logika pengelolaan data dari logika tampilan, meningkatkan keterbacaan kode, serta mempermudah pemeliharaan aplikasi dalam jangka panjang.

Pada penelitian ini, TanStack Query digunakan sebagai solusi untuk menerapkan arsitektur \textit{Client Data Layer} pada pengembangan dashboard analitik. Fokus penggunaan TanStack Query diarahkan pada pengelolaan server state, caching data, serta sinkronisasi data antar-komponen, tanpa membahas aspek internal pustaka atau detail implementasi secara mendalam. Dengan pendekatan ini, TanStack Query berperan sebagai fondasi pengelolaan data pada sisi frontend yang mendukung penyajian informasi sentimen secara konsisten, responsif, dan efisien.


\subsection{Metode Fountain}

Metode Fountain merupakan salah satu model dalam \textit{Software Development Life Cycle (SDLC)} yang bersifat iteratif dan fleksibel, sebagai alternatif terhadap model pengembangan linear seperti Waterfall. Model ini memungkinkan fase-fase pengembangan seperti analisis kebutuhan, perancangan, implementasi, dan pengujian untuk saling tumpang tindih serta dilakukan ulang sesuai kebutuhan proyek, sehingga proses pengembangan dapat menyesuaikan perubahan persyaratan yang muncul sepanjang siklus hidup sistem.

Selain bersifat iteratif, metode Fountain menekankan kelanjutan antar-tahapan pengembangan tanpa batasan transisi yang kaku. Setiap aktivitas dalam siklus pengembangan tidak dipandang sebagai tahap yang berdiri sendiri. Dengan pendekatan ini, umpan balik yang diperoleh pada tahap implementasi atau pengujian dapat segera digunakan untuk memperbaiki hasil analisis maupun perancangan sebelumnya. Hal tersebut memungkinkan kesalahan desain atau ketidaksesuaian kebutuhan terdeteksi lebih awal, sehingga risiko \textit{rework} pada tahap akhir dapat diminimalkan.

Fleksibilitas metode Fountain menjadikannya relevan untuk pengembangan sistem yang bersifat dinamis, seperti aplikasi dashboard analitik. Pada sistem semacam ini, perubahan kebutuhan pengguna, penyesuaian arsitektur sering kali muncul selama proses pengembangan berlangsung. Oleh karena itu, model Fountain memberikan kerangka kerja yang adaptif namun tetap terstruktur, sehingga pengembangan sistem dapat berjalan secara terkendali tanpa kehilangan kemampuan untuk beradaptasi terhadap perubahan yang terjadi.

\begin{figure}[H]
  \centering
  \includegraphics[width=0.3\textwidth]{gambar/fountain-sdlc.jpg}
  \caption{Fountain SDLC Model}
  \label{fig:fountain-sdlc}
\end{figure}
\FloatBarrier

Gambar \ref{fig:fountain-sdlc} menunjukkan ilustrasi model Fountain, di mana fase-fase pengembangan tidak lagi bergerak secara kaku dari satu tahap ke tahap berikutnya, tetapi fase dapat saling berinteraksi dan kembali ke fase sebelumnya bila terdapat kebutuhan penyesuaian. Karakteristik ini memungkinkan proses pengembangan sistem menjadi lebih adaptif terhadap perubahan yang tidak terduga, sehingga risikonya dapat diminimalkan dan kualitas akhir sistem meningkat.

Fountain banyak digunakan pada proyek yang memerlukan peninjauan ulang fase secara berkala, terutama ketika kebutuhan belum sepenuhnya jelas di awal atau kemungkinan perubahan tinggi. Dengan mekanisme iteratif dan fleksibel, model ini mendukung evaluasi berkelanjutan terhadap artefak pengembangan tanpa mengganggu keseluruhan proses pengembangan sistem.

Menurut \citep{Fadillah2023}, metode Fountain banyak digunakan pada pengembangan perangkat lunak dengan kompleksitas menengah hingga tinggi, khususnya pada sistem yang kebutuhan fungsionalnya dapat berkembang seiring berjalannya proses implementasi. Dengan sifatnya yang fleksibel dan iteratif, metode ini dinilai sesuai untuk penelitian rekayasa perangkat lunak yang membutuhkan penyesuaian desain dan pengambilan keputusan teknis secara berulang tanpa mengganggu keseluruhan alur pengembangan sistem .

\subsubsection{Tahapan Metode Fountain}
\vspace*{-1cm}

Metode Fountain terdiri dari beberapa tahapan pengembangan perangkat lunak yang dilakukan secara iteratif dan fleksibel. Setiap tahapan tidak bersifat kaku dan dapat saling tumpang tindih, sehingga memungkinkan penyesuaian kembali ke tahapan sebelumnya apabila ditemukan perubahan kebutuhan atau permasalahan selama proses pengembangan. Secara umum, tahapan dalam metode Fountain meliputi analisis, requirement specification, design, coding, testing, operation, maintenance, dan Evolution \cite{Fadillah2023}.

Tahap \textit{analysis} merupakan tahapan awal yang bertujuan untuk memahami permasalahan dan kebutuhan sistem secara umum. Pada tahap ini dilakukan identifikasi kebutuhan pengguna, ruang lingkup sistem, serta permasalahan yang ingin diselesaikan melalui pengembangan perangkat lunak. Hasil dari tahap analisis menjadi dasar bagi tahapan-tahapan selanjutnya.

Tahap \textit{requirement specification} berfokus pada perumusan kebutuhan sistem secara lebih terstruktur dan terdokumentasi. Kebutuhan tersebut mencakup kebutuhan fungsional dan non-fungsional yang harus dipenuhi oleh sistem. Spesifikasi kebutuhan berperan penting sebagai acuan dalam proses perancangan dan implementasi sistem.

Tahap \textit{design} merupakan tahapan perancangan solusi berdasarkan spesifikasi kebutuhan yang telah ditetapkan. Pada tahap ini dilakukan perancangan arsitektur sistem, struktur data, serta rancangan antarmuka pengguna. Hasil perancangan bertujuan untuk memberikan gambaran teknis mengenai bagaimana sistem akan dibangun sebelum masuk ke tahap implementasi.

Tahap \textit{coding} adalah tahapan implementasi dari desain yang telah dibuat ke dalam bentuk kode program. Pada tahap ini, pengembang menerjemahkan rancangan sistem menjadi perangkat lunak yang dapat dijalankan sesuai dengan kebutuhan yang telah ditentukan.

Tahap \textit{testing} dilakukan untuk memastikan bahwa sistem yang dikembangkan telah berjalan sesuai dengan spesifikasi dan bebas dari kesalahan fungsional. Pengujian dilakukan untuk memverifikasi fungsi-fungsi sistem serta memastikan bahwa sistem dapat digunakan dengan baik oleh pengguna.

Tahap \textit{operation} merupakan tahapan di mana sistem telah siap digunakan dalam lingkungan operasional. Pada tahap ini, sistem mulai diakses oleh pengguna dan digunakan untuk mendukung aktivitas yang telah dirancang.

Tahap \textit{maintenance} bertujuan untuk menjaga kinerja sistem setelah digunakan secara operasional. Aktivitas pada tahap ini meliputi perbaikan kesalahan, penyesuaian terhadap perubahan lingkungan, serta peningkatan minor terhadap sistem agar tetap berjalan dengan optimal.

Tahap \textit{Evolution} merupakan tahapan pengembangan lanjutan dalam metode Fountain yang bertujuan untuk melakukan peningkatan dan penambahan fitur pada sistem berdasarkan hasil penggunaan dan kebutuhan yang muncul. Tahapan ini memastikan sistem dapat terus dikembangkan secara berkelanjutan agar tetap sesuai dengan kebutuhan pengguna dan tujuan pengembangan
% \vspace{-.2cm}
\subsubsection{Alasan Pemilihan Metode Fountain}
% \vspace{-.3cm}
Pemilihan metode \textit{Software Development Life Cycle (SDLC)} yang tepat merupakan langkah strategis yang krusial untuk memastikan keberhasilan proyek serta mencegah pembengkakan biaya dan waktu pengembangan \citep{Aniley2024}. Berdasarkan pertimbangan tersebut, penelitian ini mengadopsi metode Fountain yang dinilai relevan karena memiliki karakteristik fleksibel namun tetap terstruktur. Efektivitas metode ini didukung oleh penelitian \citep{Sastra2023} pada perancangan sistem SIP-PTK, yang menunjukkan bahwa model Fountain mampu memandu tahapan analisis, desain, implementasi, hingga pengujian secara efektif pada sistem yang memiliki kebutuhan pengolahan data yang spesifik.

Relevansi metode Fountain juga sejalan dengan fokus penelitian ini, yaitu pengembangan aplikasi frontend yang bersifat data-driven dan memiliki ketergantungan tinggi terhadap interaksi antar-komponen. Dalam pengembangan frontend, tahapan perancangan arsitektur, implementasi, dan pengujian tidak berjalan secara linier, melainkan saling berkaitan dan sering memerlukan penyesuaian berdasarkan hasil evaluasi sistem. Penerapan metode Fountain memungkinkan tahapan pengembangan dan evaluasi berjalan secara paralel serta saling tumpang tindih (overlapping) tanpa menghilangkan struktur dan urutan penelitian yang jelas.

Melalui pendekatan ini, aktivitas analisis kebutuhan, perancangan arsitektur frontend, implementasi Client Data Layer, serta pengujian sistem dapat dilakukan secara berulang dan saling mempengaruhi. Hasil evaluasi pada satu tahapan dapat langsung digunakan sebagai dasar penyesuaian pada tahapan lainnya, seperti perubahan struktur data atau mekanisme sinkronisasi, tanpa harus menunggu seluruh siklus pengembangan selesai. Dengan demikian, proses pengembangan sistem menjadi lebih adaptif dan terkontrol, sehingga diharapkan mampu menghasilkan hasil penelitian yang sesuai dengan tujuan penelitian secara optimal.


\section{Penelitian Terkait}

Berikut adalah tabel perbandingan penelitian terkait yang relevan dengan pengembangan penerapan arsitektur \textit{Client Data Layer} menggunakan TanStack Query pada dashboard analisis sentimen
\setlength{\tabcolsep}{6pt}
\renewcommand{\arraystretch}{1.3}

\begin{longtable}{|c|p{3cm}|p{3cm}|p{3cm}|p{4cm}|}
\caption{Tabel Perbandingan Penelitian Terkait}
\label{tab:perbandingan_penelitian} \\

\hline
\textbf{No} &
\textbf{Peneliti} &
\textbf{Teknologi} &
\textbf{Judul} &
\textbf{Fitur} \\ \hline
\endfirsthead

\hline
\textbf{No} &
\textbf{Peneliti} &
\textbf{Teknologi} &
\textbf{Judul} &
\textbf{Fitur} \\ \hline
\endhead

\hline
\endfoot

\hline
\endlastfoot

1 &
Fajarini, Sri Dwi; Kurniawati, Juliana; Yuliani, Fitria (2025) &
NLP, Machine Learning (SVM, Random Forest, VADER) &
\textit{Social Media Sentiment Analysis as a New Tool for Predicting Market Trends and Consumer Behaviour} &
Analisis sentimen media sosial untuk mengidentifikasi pola opini publik dan memprediksi perilaku konsumen serta tren pasar. \\ \hline

2 &
Shrutika Rathore; Rahul Nawkhare; Navin Sharma; Nitin Chaudhary; Saurabh Chakole; Bhaskar Vishwakrama (2025) &
Dashboard analitik, visualisasi data, business intelligence &
\textit{Effective Data Visualization Techniques for Business Decision-Makers} &
Penyajian data bisnis melalui dashboard analitik untuk meningkatkan efisiensi pengambilan keputusan dan perencanaan strategis. \\ \hline

3 &
Micheal, Dave (2024) &
React, TanStack Query, lazy loading, caching &
\textit{React Query and Lazy Loading: Performance Optimization Best Practices} &
Optimasi performa aplikasi frontend melalui pengelolaan data asinkron, caching, dan lazy loading untuk mengurangi permintaan API berulang. \\ \hline

\end{longtable}

Berdasarkan kajian terhadap jurnal-jurnal penelitian terdahulu, dapat diidentifikasi adanya celah gap penelitian yang berkaitan dengan pemanfaatan hasil analisis sentimen media sosial pada sisi frontend aplikasi dashboard analitik. Sejumlah penelitian berfokus pada penerapan analisis sentimen menggunakan pendekatan Natural Language Processing (NLP) dan machine learning untuk memprediksi perilaku konsumen serta tren pasar. Namun demikian, penelitian-penelitian tersebut umumnya menempatkan analisis sentimen sebagai fokus utama dan belum membahas secara mendalam bagaimana hasil analisis sentimen tersebut dikelola dan disajikan pada sisi frontend, khususnya dalam bentuk dashboard analitik yang digunakan secara langsung oleh pengguna.

Selain itu, penelitian lain menekankan pada peran dashboard dan teknik visualisasi data dalam meningkatkan efektivitas pengambilan keputusan bisnis. Fokus kajian lebih diarahkan pada desain visualisasi, jenis grafik, serta manfaat dashboard sebagai alat bantu analisis. Akan tetapi, aspek teknis pengelolaan data pada sisi frontend, seperti arsitektur pengambilan data dari API, pengelolaan server state, mekanisme caching, serta konsistensi data antar-komponen pada aplikasi frontend modern, masih belum menjadi perhatian utama dalam penelitian-penelitian tersebut.

Di sisi lain, terdapat penelitian yang membahas penggunaan React Query dalam pengelolaan data asinkron pada aplikasi React, termasuk pemanfaatan caching dan lazy loading untuk meningkatkan performa aplikasi frontend. Meskipun penelitian ini menunjukkan efektivitas React Query dalam mengurangi pemanggilan API berulang, konteks penerapannya masih bersifat umum dan belum secara spesifik dikaji sebagai bagian dari arsitektur Client Data Layer pada aplikasi dashboard analisis sentimen. Selain itu, keterkaitan antara pengelolaan server state di sisi frontend dengan kebutuhan penyajian data analitik pada konteks UMKM juga belum banyak dibahas. Oleh karena itu, penelitian ini mengisi celah tersebut dengan mengkaji penerapan arsitektur Client Data Layer menggunakan TanStack Query pada pengembangan Dashboard Analisis Sentimen UMKM.
