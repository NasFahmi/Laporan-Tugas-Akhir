%==================================================================
% Ini adalah abstrak dalam bahasa indonesia 
%==================================================================

%% DILARANG EDIT BAGIAN INI
\clearpage
\phantomsection
\addcontentsline{toc}{chapter}{ABSTRAK}
\begin{center}
    \textbf{\large{\judulid}}\\[0.5cm]
    Oleh\\
    \penulis\\
    NIM: \nim\\[2em]
    \textbf{ABSTRAK}\\[0.5cm]
\end{center}
%% DILARANG EDIT BAGIAN INI
Di era digital, pelaku UMKM semakin bergantung pada media sosial seperti Instagram untuk memahami persepsi publik, di mana komentar pengguna menghasilkan data opini yang bersifat tidak terstruktur dan dinamis. Data tersebut umumnya diolah menggunakan \textit{Sentiment Analysis} berbasis \textit{Natural Language Processing (NLP)} untuk mengekstraksi data opini konsumen. Hasil analisis sentimen selanjutnya disajikan dalam bentuk dashboard analitik berbasis web guna mendukung pemantauan dan pengambilan keputusan. Namun, pengembangan dashboard menghadapi tantangan dalam pengelolaan alur data di sisi frontend, seperti permintaan API berulang dan inkonsistensi data antar-komponen. Oleh karena itu, penelitian ini bertujuan menerapkan arsitektur Client Data Layer menggunakan pustaka TanStack Query untuk menciptakan pengelolaan data frontend yang lebih terstruktur serta menganalisis perilaku sistem terkait konsistensi data dan efisiensi pengambilan data melalui mekanisme caching.

Pengembangan sistem ini dilakukan menggunakan metode Fountain yang bersifat iteratif dan fleksibel, memungkinkan fase analisis, perancangan, implementasi, dan pengujian berjalan secara tumpang tindih. Pada sisi teknis, sistem dibangun menggunakan framework React dengan TanStack Query sebagai \textit{server state management}. Testing sistem dilakukan menggunakan pendekatan black-box testing berbasis skenario untuk mengevaluasi perilaku sistem frontend dalam mengelola \textit{server state}, termasuk konsistensi data antar-komponen, mekanisme caching, dan pengendalian pemanggilan API.

Hasil penelitian menunjukkan bahwa penerapan arsitektur \textit{Client Data Layer} berhasil menjaga konsistensi informasi di seluruh bagian dashboard tanpa memicu permintaan API tambahan saat pengguna melakukan navigasi antar-halaman. Pengujian fungsional terhadap 37 skenario menunjukkan tingkat keberhasilan yang tinggi, meskipun ditemukan satu kegagalan minor pada representasi state awal pemuatan data. Dapat disimpulkan bahwa penggunaan TanStack Query secara signifikan meningkatkan efisiensi komunikasi antara frontend dan backend serta mendukung skalabilitas aplikasi analitik yang bersifat data-driven. Sebagai saran, integrasi dengan pustaka state management lokal seperti Zustand dapat dipertimbangkan untuk mengoptimalkan pengelolaan client state yang tidak bergantung pada server \\ [0.6cm]
%% DILARANG EDIT BAGIAN INI
\noindent Kata kunci: \textit{Sentiment Analysis}, \textit{Client Data Layer}, TanStack Query, Metode Fountain, Dashboard Analitik
%% DILARANG EDIT BAGIAN INI