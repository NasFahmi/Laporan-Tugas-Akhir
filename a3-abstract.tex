%==================================================================
% Ini adalah abstrak dalam bahasa inggris 
%==================================================================

%% DILARANG EDIT BAGIAN INI
\clearpage
\phantomsection
\addcontentsline{toc}{chapter}{ABSTRACT}
\begin{center}
    \textbf{\large{\judulen}}
\end{center}

\vspace{0.5cm}

\noindent
\begin{tabular}{@{}ll}
    Name       & : \penulis \\
    NIM        & : \nim \\
    Advisors   & : \begin{tabular}[t]{@{}l@{}}
                      $1$ \pembimbingsatu \\
                      $2$ \pembimbingdua
                   \end{tabular}
\end{tabular}

\vspace{1cm}

\begin{center}
    \textbf{ABSTRACT}
\end{center}

\vspace{0.5cm}

%% edit bagian ini
\textit{In the digital era, Micro, Small, and Medium Enterprises (MSMEs) increasingly rely on social media platforms such as Instagram to understand public perception through unstructured and dynamic user comment data. These data are commonly processed using Natural Language Processing (NLP)-based sentiment analysis and presented through web-based analytical dashboards to support monitoring and decision-making. However, the development of such dashboards introduces challenges in frontend data flow management, including repeated API requests and data inconsistency across components. Therefore, this study aims to implement a Client Data Layer architecture using the TanStack Query library and to analyze frontend system behavior in terms of data consistency and data retrieval efficiency through caching mechanisms.}\\

\textit{This research was conducted using the Fountain method, which supports iterative and overlapping phases of analysis, design, implementation, and testing. The system was developed as a React-based frontend application, with TanStack Query employed as a server state management solution to handle data fetching, temporary storage, and synchronization from a backend REST API. System evaluation was carried out using a scenario-based black-box testing approach to examine frontend data management behavior, including data consistency between components, caching effectiveness, and API request control under various usage scenarios.}\\

\textit{The results indicate that the implementation of the Client Data Layer architecture using TanStack Query successfully maintains information consistency across all dashboard components without triggering redundant API requests during page navigation. Testing across 37 functional scenarios demonstrated a high success rate, with only one minor issue identified in the initial data loading state representation. Overall, this study concludes that the use of TanStack Query effectively improves frontend–backend communication efficiency and supports the development of more structured, stable, and scalable data-driven analytical dashboards.}\\[0.6cm]
%% edit sampai sini

%% DILARANG EDIT BAGIAN INI
\noindent Key words: Client Data Layer, TanStack Query, Analytical Dashboard, Data-Driven Frontend, Server State Management
%% DILARANG EDIT BAGIAN INI