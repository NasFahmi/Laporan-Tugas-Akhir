%==================================================================
% Ini adalah abstrak dalam bahasa inggris 
%==================================================================

%% DILARANG EDIT BAGIAN INI
\clearpage
\phantomsection
\addcontentsline{toc}{chapter}{ABSTRACT}
\begin{center}
    \textbf{\large{\judulen}}\\[0.5cm]
    by:\\
    \penulis\\
    NIM: \nim\\[2em]
    \textbf{ABSTRACT}\\[0.5cm]
\end{center}
%% DILARANG EDIT BAGIAN INI

%% edit bagian ini
In the digital era, Micro, Small, and Medium Enterprises (MSMEs) increasingly rely on social media platforms such as Instagram to understand public perception, where user comments generate large volumes of unstructured and dynamic opinion data. These data are commonly processed using sentiment analysis based on Natural Language Processing (NLP) to extract consumer sentiment information. The resulting sentiment data are subsequently presented through a web-based analytical dashboard to support monitoring and data-driven decision-making. However, the development of data-driven dashboards introduces challenges in frontend data management, particularly related to repeated API requests and data inconsistency across interface components. Therefore, this research aims to implement a Client Data Layer architecture using the TanStack Query library and to analyze frontend system behavior in terms of data consistency and data retrieval efficiency through caching mechanisms.

The system was developed using the Fountain method, which supports iterative and overlapping phases of analysis, design, implementation, and testing. From a technical perspective, the frontend application was built using the React framework, with TanStack Query utilized as a server state management solution. System evaluation was conducted using a scenario-based black-box testing approach to observe frontend behavior in managing server state, including data consistency across components, caching effectiveness, and API request control.

The results indicate that the implementation of the Client Data Layer architecture successfully maintains data consistency throughout the dashboard without triggering redundant API requests during page navigation. Functional testing across 37 scenarios demonstrated a high success rate, with only a minor issue identified in the initial loading state representation. It can be concluded that the use of TanStack Query improves frontend–backend communication efficiency and supports the scalability of data-driven analytical dashboards. Future work may consider integrating local state management libraries such as Zustand to further optimize client-side state handling that is independent of server data\\[0.6cm]
%% edit sampai sini

%% DILARANG EDIT BAGIAN INI
\noindent Key words: Abstract Concepts, Abstract Components, Key Words.
%% DILARANG EDIT BAGIAN INI