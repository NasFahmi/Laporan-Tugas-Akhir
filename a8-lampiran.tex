%==================================================================
% Ini adalah lampiran
%==================================================================
%% DILARANG EDIT BAGIAN INI
\appendix
\chapter*{LAMPIRAN}
\addcontentsline{toc}{chapter}{LAMPIRAN}
%% DILARANG EDIT BAGIAN INI

% Definisi counter untuk lampiran agar ref berfungsi
\newcounter{lampiran}
\renewcommand{\thelampiran}{Lampiran \arabic{lampiran}}

\phantomsection
\refstepcounter{lampiran}
\section*{\thelampiran. Kode Program \textit{Query Client}}
\addcontentsline{toc}{section}{\thelampiran. Kode Program \textit{Query Client}}
\label{app:query-client} 
\lstinputlisting{kode/query-client.ts}

\hfill

\phantomsection
\refstepcounter{lampiran}
\section*{\thelampiran. Kode Program \textit{Query Keys}}
\addcontentsline{toc}{section}{\thelampiran. Kode Program \textit{Query Keys}}
\label{app:query-keys} 
\lstinputlisting{kode/query-keys.ts}

\hfill

\phantomsection
\refstepcounter{lampiran}
\section*{\thelampiran. Kode Program \textit{Interface}}
\addcontentsline{toc}{section}{\thelampiran. Kode Program \textit{Interface}}
\label{app:interface}  
\lstinputlisting{kode/scraper.ts}


\hfill

\phantomsection
\refstepcounter{lampiran}
\section*{\thelampiran. Kode Program \textit{Repository}}
\addcontentsline{toc}{section}{\thelampiran. Kode Program \textit{Repository}}
\label{app:repository} 
\lstinputlisting{kode/scraper.repository.tsx}

\hfill

\phantomsection
\refstepcounter{lampiran}
\section*{\thelampiran. Kode Program \textit{Query}}
\addcontentsline{toc}{section}{\thelampiran. Kode Program \textit{Query}}
\label{app:query} 
\lstinputlisting{kode/useScraperQuery.ts}

\hfill
\phantomsection
\refstepcounter{lampiran}
\section*{\thelampiran. Kode Program \textit{Mutation}}
\addcontentsline{toc}{section}{\thelampiran. Kode Program \textit{Mutation}}
\label{app:mutation} 
\lstinputlisting{kode/useLoginMutation.ts}

\hfill
% \clearpage
\phantomsection
\refstepcounter{lampiran}
\section*{\thelampiran. Dokumentasi Bimbingan}
\addcontentsline{toc}{section}{\thelampiran. Dokumentasi Bimbingan}
\begin{table}[H]
	\centering
	\begin{tabular}{cc}
		\includegraphics[width=0.4\textwidth]{gambar/bimbingan/b-arum01.jpeg} & 
		\includegraphics[width=0.35\textwidth]{gambar/bimbingan/b-arum02.jpeg} \\
		(a) Dokumentasi Bimbingan 1 & (b) Dokumentasi Bimbingan 2 \\[1em]
		\includegraphics[width=0.4\textwidth]{gambar/bimbingan/pak-sep-1.jpeg} & 
		\includegraphics[width=0.4\textwidth]{gambar/bimbingan/pak-sep-2.jpeg} \\
		(c) Dokumentasi Bimbingan 3 & (d) Dokumentasi Bimbingan 4 \\
	\end{tabular}
\end{table}

\hfill
\phantomsection
\refstepcounter{lampiran}
\section*{\thelampiran. Dokumentasi Penyerahan Hadiah}
\addcontentsline{toc}{section}{\thelampiran. Dokumentasi Penyerahan Hadiah}
\begin{figure}[H]
\centering
\includegraphics[width=0.7\textwidth]{gambar/gambar-penyerahan-hadiah.jpeg}
\end{figure}

\hfill
\phantomsection
\refstepcounter{lampiran}
\section*{\thelampiran. Dokumentasi dengan Dosen Pembimbing}
\addcontentsline{toc}{section}{\thelampiran. Dokumentasi dengan Dosen Pembimbing}
\begin{figure}[H]
\centering
\includegraphics[width=0.7\textwidth]{gambar/gambar-full-team-dospem.jpeg}
\end{figure}


\hfill
\phantomsection
\refstepcounter{lampiran}
\section*{\thelampiran. Sertifikat}
\addcontentsline{toc}{section}{\thelampiran. Sertifikat}
\begin{figure}[H]
\centering
\includegraphics[width=0.7\textwidth]{gambar/sertifikat.jpg}
\end{figure}