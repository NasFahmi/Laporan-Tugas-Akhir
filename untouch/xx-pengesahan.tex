%==================================================================
% HALAMAN PENGESAHAN – FIXED
%==================================================================

\newpage
\addcontentsline{toc}{chapter}{HALAMAN PENGESAHAN}
\thispagestyle{plain}
\raggedbottom
\enlargethispage{2\baselineskip}

\begin{center}
\singlespacing
\fontsize{12pt}{14.5pt}\selectfont

% ===== JUDUL =====
\textbf{\MakeUppercase{\judulid}}
\end{center}

\vspace{0.8\baselineskip}

\begin{center}
{\singlespacing\normalsize
\textbf{Tugas Akhir Ini Disusun Untuk Memenuhi Salah Satu Syarat Memperoleh}\\
\textbf{Gelar Sarjana Terapan Komputer (S.Tr.Kom) \linebreak \universitas}\\
}
\end{center}


\vspace{1\baselineskip}

\begin{center}
\textbf{oleh:}\\
\textbf{\MakeUppercase{\penulis}}\\
\textbf{NIM \nim}
\end{center}

\vspace{1\baselineskip}

\begin{center}
\textbf{Tanggal Ujian : \tglpengesahan}\\
\vspace{0.5\baselineskip}
\textbf{Menyetujui,}
\end{center}

\vspace{1\baselineskip}

% ===== DAFTAR PENGUJI & PEMBIMBING (DENGAN TANDA TANGAN) =====
\noindent
\begin{table}[h!]
    \begin{tabular}{lll}
        \textbf{Pembimbing 1} & : \textbf{\pembimbingsatu}                           & (.......................)                                          \\[1cm]
        \textbf{Pembimbing 2} & : \textbf{\pembimbingdua}                            & (.......................)                                          \\[1cm]
        \textbf{Penguji 1} & : \textbf{\pengujisatu}                              & (.......................)                                          \\[1cm]
        \textbf{Penguji 2} & : \textbf{\pengujidua}                               & (.......................)                                          \\[1cm]
    \end{tabular}
\end{table}


\begin{singlespace}
\noindent
\begin{tabular}{@{}p{6cm} p{1cm} p{8cm}@{}}
\textbf{Mengesahkan,} & & \textbf{Mengetahui,} \\[0.3\baselineskip]
\textbf{Ketua Jurusan} & & \textbf{Koordinator Program Studi} \\[0.3\baselineskip]
\textbf{Bisnis dan Informatika} & & \textbf{Teknologi Rekayasa Perangkat Lunak} \\[2cm]

\underline{\textbf{\dekan}} & & \underline{\textbf{\koorprodi}} \\[0.1\baselineskip] 
\textbf{NIP. \NIPdekan} & & \textbf{NIP. \NIPkoorprodi}
\end{tabular}
\end{singlespace}

