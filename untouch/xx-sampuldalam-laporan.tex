%==================================================================
% Ini adalah sampul dalam
%==================================================================


\newpage
\addcontentsline{toc}{chapter}{HALAMAN SAMPUL}

% Makro jarak (berbasis spasi aktif)
\newcommand{\coverspace}{\vspace{1.5\baselineskip}}

\begin{center}
\onehalfspacing
\fontsize{12pt}{14.5pt}\selectfont

    % ===== Judul =====
    {\normalsize\bfseries\MakeUppercase{\judulid}}

    \coverspace\coverspace

    % ===== Jenis laporan =====
    {\normalsize\bfseries\MakeUppercase{LAPORAN \tipe}}\\
    {\normalsize\bfseries\MakeUppercase{PENGHARGAAN SKEMA JAWA TIMUR DATATHON}}

    \coverspace\coverspace

    % ===== Logo =====
    \includegraphics[width=4cm,height=4cm,keepaspectratio]{gambar/logo-poliwangi.png}

    \coverspace\coverspace

    % ===== Keterangan =====
    Tugas Akhir ini Dibuat dan Diajukan untuk Memenuhi Salah Satu Syarat Kelulusan\\
    Program Studi Sarjana Terapan \prodi\\
    dan Mencapai Gelar Sarjana Terapan Komputer (S.Tr.Kom)

    \coverspace\coverspace

    % ===== Penulis =====
  % ===== Penulis =====
\textbf{Oleh:}\\
\textbf{\MakeUppercase{\penulis}}\\
\textbf{NIM \nim}

\vfill


\vfill


\begin{onehalfspace}

  \fontsize{12pt}{14.5pt}\selectfont
  % ===== Institusi =====
  \textbf{\MakeUppercase{Program Studi Sarjana Terapan}}\\
  \textbf{\MakeUppercase{\prodi}}\\
  \textbf{\MakeUppercase{\jurusan}}\\
  \textbf{\MakeUppercase{\universitas}}\\
  \textbf{\the\year}
\end{onehalfspace}


\end{center}