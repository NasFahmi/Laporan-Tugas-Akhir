%==================================================================
% Ini adalah bab 4
% Silahkan edit sesuai kebutuhan, baik menambah atau mengurangi \section, \subsection
%==================================================================

\chapter[HASIL DAN PEMBAHASAN]{\\ HASIL DAN PEMBAHASAN}

\section{Hasil}

\subsection{\textit{Analysis}}
Pada tahap \textit{analysis}, hasil yang diperoleh berupa identifikasi kebutuhan sistem frontend dashboard analisis sentimen UMKM yang bersifat data-driven. Analisis menunjukkan bahwa data hasil analisis sentimen yang bersumber \textit{service} ABSA (\textit{Aspect Based Sentiment Analysis}) dan Rekomendasi Konten yang di salurkan ke backend digunakan oleh beberapa komponen antarmuka secara bersamaan, seperti dashboard utama, halaman sentiment, dan halaman rekomendasi. Kondisi ini menimbulkan kebutuhan akan mekanisme pengelolaan data frontend yang mampu menjaga konsistensi data serta menghindari permintaan data berulang ke REST API.

Hasil analisis juga menunjukkan bahwa pengguna sistem hanya berperan sebagai konsumen informasi, sehingga kebutuhan utama sistem terletak pada stabilitas penyajian data, konsistensi antar-komponen, serta kemudahan navigasi tanpa kehilangan data. Temuan pada tahap ini menjadi dasar pemilihan arsitektur Client Data Layer sebagai solusi pengelolaan data frontend.

\subsection{\textit{Requirement Specification}}
Hasil dari tahap requirement specification adalah tersusunnya kebutuhan fungsional dan non-fungsional sistem frontend. Kebutuhan fungsional mencakup kemampuan sistem untuk menampilkan ringkasan sentimen, visualisasi distribusi sentimen, data rekomendasi, serta data scraper yang bersumber dari REST API backend.

Kebutuhan non-fungsional yang dihasilkan pada tahap ini meliputi konsistensi data antar-komponen, pengendalian permintaan API, serta kemampuan sistem dalam memanfaatkan mekanisme caching. Kebutuhan tersebut menjadi acuan dalam perancangan arsitektur frontend dan pemilihan TanStack Query sebagai pustaka pengelolaan Client Data Layer.

\subsection{\textit{Design}}

Pada tahap design, dihasilkan rancangan arsitektur frontend berbasis component-based architecture dengan pemisahan antara lapisan presentasi dan lapisan pengelolaan data. Hasil perancangan menunjukkan bahwa Client Data Layer diposisikan sebagai lapisan perantara antara REST API dan komponen antarmuka pengguna.

Selain perancangan arsitektur, tahap ini juga menghasilkan rancangan penggunaan TanStack Query sebagai implementasi Client Data Layer, serta Desain antarmuka untuk setiap halaman utama sistem. Rancangan tersebut menjadi acuan utama dalam proses implementasi frontend pada tahap selanjutnya.

\subsubsection{Arsitektur \textit{Frontend}}

Arsitektur frontend pada sistem yang diimplementasikan juga ditunjukkan melalui cara penulisan dan pengelompokan file di dalam proyek frontend yang mengikuti arsitektur yang diterapkan. Struktur frontend disusun dengan pemisahan yang jelas antara komponen \textit{view}, logika pengelolaan data, serta definisi \textit{type} atau struktur data dari \textit{response} maupun \textit{payload}, dan aturan validasi atau \textit{schema} yang digunakan sistem. Pemisahan ini bertujuan untuk mendukung penerapan arsitektur frontend yang terstruktur dan mudah dipelihara.

Komponen \textit{view} berperan sebagai lapisan antarmuka pengguna yang bertanggung jawab dalam menampilkan data dan menangani interaksi pengguna dengan sistem. Lapisan ini tidak menangani proses pengambilan data secara langsung, melainkan hanya menerima data yang telah dikelola oleh lapisan pengelolaan data. Dengan pendekatan ini, perubahan pada logika data tidak berdampak langsung terhadap tampilan antarmuka pengguna.

Lapisan logika pengelolaan data bertugas mengatur proses komunikasi dengan REST API, termasuk pengambilan data, pengelolaan status data, serta distribusi data ke komponen \textit{view}. Lapisan ini menjadi perantara antara sumber data eksternal dan antarmuka pengguna, sehingga seluruh pengelolaan data dapat dilakukan secara terpusat dan konsisten.

Definisi \textit{type} atau struktur data digunakan untuk merepresentasikan bentuk data yang dipertukarkan antara frontend dan backend, baik dalam bentuk \textit{payload} permintaan maupun \textit{response} dari server. Aturan validasi atau \textit{schema} berfungsi untuk memastikan data yang diproses dan dikirimkan oleh sistem telah sesuai dengan format dan ketentuan yang ditetapkan. Pemisahan definisi struktur data dan aturan validasi ini membantu menjaga konsistensi data serta meminimalkan kesalahan pada proses pengolahan data.

\subsubsection{\textit{Client Data Layer}}

Hasil implementasi menunjukkan bahwa pengelolaan data pada sistem frontend berjalan secara konsisten pada seluruh halaman dashboard. Pada saat data pertama kali diakses oleh aplikasi, sistem mengambil data dari REST API dan menyimpannya sebagai bagian dari pengelolaan data internal. Data yang telah diperoleh tersebut selanjutnya digunakan kembali oleh fitur-fitur lain yang membutuhkan sumber data yang sama tanpa memicu permintaan ulang ke server. Alur pengelolaan data ini ditunjukkan pada Gambar~\ref{fig:tanstack-works}.

Penerapan pengelolaan data terpusat ini memengaruhi perilaku sistem pada saat aplikasi digunakan. Ketika pengguna melakukan navigasi antar-halaman dan kembali ke halaman sebelumnya, data tetap ditampilkan tanpa mengalami pemuatan ulang. Perilaku ini teramati secara konsisten pada halaman dashboard utama dan halaman dashboard sentimen yang menggunakan sumber data yang sama. Kondisi tersebut menunjukkan bahwa data yang telah diperoleh dapat dimanfaatkan kembali selama masih relevan dengan kebutuhan sistem.

Selain itu, implementasi \textit{Client Data Layer} juga berdampak pada efisiensi permintaan data. Ketika beberapa komponen frontend membutuhkan data yang sama dalam waktu yang bersamaan, sistem tidak melakukan permintaan data secara terpisah untuk setiap komponen. Sebaliknya, satu hasil pengambilan data digunakan bersama oleh seluruh komponen terkait. Perilaku ini mengurangi permintaan data yang berulang dan mendukung efisiensi komunikasi antara frontend dan backend.

Pengelolaan data melalui \textit{Client Data Layer} juga mendukung konsistensi penyajian data antar-komponen frontend. Data yang digunakan oleh beberapa halaman atau komponen tetap berada pada kondisi yang selaras, sehingga tidak terjadi perbedaan informasi yang ditampilkan kepada pengguna. Dengan pendekatan ini, sistem frontend mampu mempertahankan konsistensi data tanpa memerlukan pengelolaan state secara manual pada setiap komponen antarmuka.

Secara keseluruhan, penerapan \textit{Client Data Layer} menggunakan TanStack Query menghasilkan perilaku pengelolaan data frontend yang lebih terstruktur, konsisten, dan efisien pada tahap implementasi sistem. Lapisan ini mendukung kebutuhan aplikasi dashboard analisis sentimen yang bersifat data-driven, di mana data yang sama digunakan oleh berbagai komponen dan perlu disajikan secara konsisten selama siklus penggunaan aplikasi.
\subsubsection{Desain Antarmuka}

Desain antarmuka pada sisi frontend dashboard analisis sentimen direalisasikan berdasarkan rancangan wireframe yang telah disusun pada tahap perancangan. Seluruh antarmuka yang diimplementasikan mencakup halaman landing page, login, register, dashboard, analisis sentimen, rekomendasi konten, dan halaman scraper. Implementasi desain antarmuka bertujuan untuk memastikan bahwa setiap fitur sistem dapat diakses dan digunakan sesuai dengan alur yang telah dirancang. Setiap halaman dirancang untuk menyajikan informasi yang relevan dengan konteks penggunaannya. Adapun hasil realisasi dari wireframe dapat dilihat pada Gambar~\ref{fig:desain-landing-page} sampai Gambar~\ref{fig:desain-scraper}.
\begin{packed_enum}
  \item Landing Page \hfill \\
  
  \begin{figure}[H]
    \centering
    \includegraphics[width=0.9\textwidth]{gambar/ui-landing-page.png}
    \caption{Desain Halaman Landing Page}
    \label{fig:desain-landing-page}
  \end{figure}
  
  Gambar~\ref{fig:desain-landing-page} menunjukkan hasil implementasi antarmuka landing page. Halaman ini berfungsi sebagai halaman awal sistem yang memberikan gambaran umum mengenai aplikasi dashboard analisis sentimen. Landing page menyediakan informasi singkat terkait fungsi sistem serta navigasi menuju halaman login dan registrasi bagi pengguna.


  \item Login \hfill \\
  
  \begin{figure}[H]
    \centering
    \includegraphics[width=0.7\textwidth]{gambar/ui-login-page.png}
    \caption{Desain Halaman Login}
    \label{fig:desain-login}
  \end{figure}

  Gambar~\ref{fig:desain-login} menunjukkan antarmuka halaman login yang digunakan untuk proses autentikasi pengguna. Pada halaman ini, pengguna dapat memasukkan kredensial berupa alamat surel dan kata sandi untuk mengakses sistem. Implementasi halaman login memastikan bahwa hanya pengguna yang terdaftar yang dapat masuk ke dalam dashboard analisis sentimen.


  \item Register \hfill \\
  
  \begin{figure}[H]
    \centering
    \includegraphics[width=0.7\textwidth]{gambar/ui-register-page.png}
    \caption{Desain Halaman Register}
    \label{fig:desain-register}
  \end{figure}

  Gambar~\ref{fig:desain-register} menampilkan hasil implementasi halaman registrasi pengguna. Halaman ini digunakan untuk proses pendaftaran pengguna baru dengan mengisi data yang diperlukan. Data yang dimasukkan pada halaman ini selanjutnya digunakan sebagai informasi akun untuk proses autentikasi pada sistem.


  \item Dashboard \hfill \\
  
  \begin{figure}[H]
    \centering
    \includegraphics[width=0.7\textwidth]{gambar/ui-dashboard-page.png}
    \caption{Desain Halaman Dashboard}
    \label{fig:desain-dashboard}
  \end{figure}

  Gambar~\ref{fig:desain-dashboard} menunjukkan antarmuka halaman dashboard utama. Halaman ini berfungsi untuk menampilkan ringkasan hasil analisis sentimen dalam bentuk visualisasi dan informasi utama. Dashboard menjadi pusat akses pengguna terhadap fitur analisis sentimen dan rekomendasi konten yang tersedia pada sistem.


  \item Sentiment \hfill \\
  
  \begin{figure}[H]
    \centering
    \includegraphics[width=0.7\textwidth]{gambar/ui-sentiment-page.png}
    \caption{Desain Halaman Sentiment}
    \label{fig:desain-sentiment}
  \end{figure}

  Gambar~\ref{fig:desain-sentiment} menampilkan antarmuka halaman analisis sentimen. Halaman ini digunakan untuk menyajikan hasil analisis sentimen secara lebih rinci berdasarkan data yang dianalisis. Informasi yang ditampilkan pada halaman ini membantu pengguna dalam memahami distribusi dan kecenderungan sentimen dari data yang diproses.


  \item Recommendation Content \hfill \\
  
  \begin{figure}[H]
    \centering
    \includegraphics[width=0.7\textwidth]{gambar/ui-recommendation-content-page.png}
    \caption{Desain Halaman Recommendation Content}
    \label{fig:desain-recommendation-content}
  \end{figure}

  Gambar~\ref{fig:desain-recommendation-content} menunjukkan antarmuka halaman rekomendasi konten. Halaman ini menyajikan hasil rekomendasi konten yang dihasilkan berdasarkan analisis sentimen data. Informasi yang ditampilkan pada halaman ini bertujuan untuk mendukung pengambilan keputusan pengguna berdasarkan hasil analisis yang tersedia.


  \item Scraper \hfill \\
  
  \begin{figure}[H]
    \centering
    \includegraphics[width=0.7\textwidth]{gambar/ui-scraper-page.png}
    \caption{Desain Halaman Scraper}
    \label{fig:desain-scraper}
  \end{figure}

  Gambar~\ref{fig:desain-scraper} menampilkan hasil implementasi halaman scraper. Halaman ini digunakan untuk melakukan proses pengambilan data dari sumber eksternal. Data yang berhasil dikumpulkan melalui halaman ini selanjutnya dapat digunakan sebagai bahan analisis pada sistem dashboard analisis sentimen.

\end{packed_enum}
\subsection{\textit{Coding/Implementation}}
\subsubsection{Landing Page}

Landing Page diimplementasikan sebagai titik awal akses pengguna ke sistem sebelum proses autentikasi dilakukan. Pada halaman ini, frontend hanya menyajikan informasi umum mengenai sistem tanpa melakukan pengambilan maupun pengolahan data dari backend, sehingga tidak melibatkan mekanisme query maupun mutation.

Implementasi Landing Page yang bersifat statis menegaskan batas antara area sistem yang tidak data-driven dan area yang memanfaatkan Client Data Layer. Dengan pemisahan ini, pengelolaan data menggunakan TanStack Query hanya aktif setelah pengguna memasuki bagian sistem yang memerlukan interaksi dengan server, sehingga struktur arsitektur frontend tetap terjaga dan tidak tercampur dengan komponen non-data.

\subsubsection{Authentication}

Fitur Authentication diimplementasikan untuk mengontrol akses pengguna terhadap sistem dashboard analisis sentimen. Proses autentikasi dilakukan melalui komunikasi frontend dengan REST API backend untuk memvalidasi kredensial pengguna sebelum mengizinkan akses ke fitur yang bersifat data-driven. Pada tahap ini, sistem mulai melibatkan pertukaran data antara frontend dan backend sebagai bagian dari alur penggunaan aplikasi.

Pada sisi frontend, hasil autentikasi menentukan status akses pengguna terhadap Client Data Layer. Pengambilan data menggunakan TanStack Query hanya dapat dilakukan setelah proses autentikasi berhasil, sehingga seluruh data dashboard berada dalam konteks pengguna yang valid. Implementasi ini memastikan bahwa pengelolaan data dan state aplikasi berjalan secara terkontrol serta mencegah akses data oleh pengguna yang tidak terautentikasi.

Hasil implementasi menunjukkan bahwa mekanisme autentikasi berfungsi sebagai gerbang awal sebelum Client Data Layer aktif digunakan. Dengan pendekatan ini, sistem frontend mampu membedakan area publik dan area terproteksi secara jelas, sekaligus menjaga konsistensi pengelolaan data pada seluruh fitur yang bergantung pada hasil autentikasi.

\subsubsection{Dashboard}
Dashboard diimplementasikan sebagai halaman ringkasan yang menyajikan sebagian informasi dari data sentimen dan rekomendasi konten. Data yang ditampilkan pada dashboard tidak diambil melalui \textit{query} tersendiri, melainkan memanfaatkan data yang telah tersedia pada \textit{Client Data Layer} dari halaman sentimen dan rekomendasi.

Pada implementasinya, dashboard berperan sebagai \textit{consumer data} yang memetakan sebagian data dari beberapa \textit{query} ke dalam komponen ringkasan. Pendekatan ini memungkinkan dashboard menampilkan informasi tanpa melakukan permintaan API tambahan, karena data telah dikelola secara terpusat oleh \textit{Client Data Layer}.

Hasil implementasi menunjukkan bahwa dashboard mampu menyajikan data yang konsisten dengan halaman sentimen dan rekomendasi. Perubahan data pada salah satu \textit{query} secara otomatis direfleksikan pada dashboard, yang menandakan bahwa pengelolaan \textit{server state} pada frontend berjalan secara terintegrasi dan efisien.
\subsubsection{Scraper}

Halaman Scraper diimplementasikan untuk menampilkan daftar data hasil proses pengambilan data media sosial yang dilakukan di sisi backend. Pada halaman ini, frontend berperan sebagai konsumen data yang bersifat informatif dan tidak melakukan pemrosesan lanjutan terhadap data yang diterima.

Implementasi halaman Scraper hanya melibatkan satu \textit{query} aktif yang dikelola oleh Client Data Layer untuk mengambil daftar data scraping. Pendekatan ini memastikan bahwa halaman Scraper tidak memicu pengambilan data yang tidak relevan serta tetap terpisah dari \textit{query} milik fitur lain.

Hasil implementasi menunjukkan bahwa pengelolaan data pada halaman Scraper berjalan secara terkontrol, di mana hanya \textit{query} yang sesuai dengan konteks halaman yang diaktifkan. Hal ini menegaskan bahwa Client Data Layer mampu membatasi ruang lingkup pengambilan data sesuai kebutuhan fitur, sehingga arsitektur frontend tetap tersegmentasi dengan baik.

\subsubsection{Sentiment}

Halaman Sentiment diimplementasikan untuk menyajikan hasil analisis sentimen secara rinci berdasarkan data yang diperoleh dari backend. Data sentimen dikelola melalui Client Data Layer dan berfungsi sebagai satu sumber data yang dipercaya bagi fitur yang membutuhkan informasi sentimen, termasuk dashboard yang menampilkan ringkasan data tersebut.

Pada implementasinya, halaman Sentiment berperan sebagai consumer utama data sentimen, di mana satu query digunakan untuk mengambil dan mengelola data sentimen secara terpusat. Data yang diperoleh kemudian dipetakan ke berbagai komponen visualisasi tanpa melibatkan pengambilan data tambahan. Pendekatan ini memastikan bahwa seluruh tampilan yang memanfaatkan data sentimen mengacu pada sumber data yang sama.

Hasil implementasi menunjukkan bahwa data sentimen yang ditampilkan pada halaman Sentiment dan dashboard selalu konsisten. Setiap perubahan data pada Client Data Layer secara otomatis direfleksikan pada kedua halaman tersebut, yang menandakan bahwa pengelolaan server state pada frontend berjalan dengan baik dan mendukung penggunaan satu sumber data yang dipercaya.

\subsubsection{Recomendation Content}

Halaman \textit{Recommendation Content} diimplementasikan untuk menyajikan rekomendasi konten yang dihasilkan berdasarkan hasil analisis sentimen. Data rekomendasi diperoleh dari backend dan dikelola melalui Client Data Layer sebelum ditampilkan pada antarmuka pengguna. Dengan demikian, fitur ini memanfaatkan hasil analisis sebagai dasar penyusunan rekomendasi yang relevan.

Pada implementasinya, halaman \textit{Recommendation Content} berperan sebagai consumer data rekomendasi, di mana satu query digunakan untuk mengelola pengambilan dan penyajian data secara terpusat. Data yang diperoleh tidak disimpan sebagai state lokal pada komponen, melainkan diakses langsung dari Client Data Layer. Pendekatan ini memastikan bahwa data rekomendasi yang digunakan bersifat konsisten dan tidak terduplikasi.

Hasil implementasi menunjukkan bahwa data rekomendasi yang ditampilkan pada halaman \textit{Recommendation Content} dan dashboard selalu selaras. Setiap pembaruan data pada Client Data Layer secara otomatis direfleksikan pada kedua halaman tersebut, yang menandakan bahwa pengelolaan data frontend berjalan secara terintegrasi dengan mengacu pada satu sumber data yang dipercaya

\subsection{\textit{Testing}}

Pengujian sistem pada penelitian ini dilakukan untuk memverifikasi bahwa seluruh fitur yang telah dirancang dan diimplementasikan berfungsi sesuai dengan kebutuhan sistem yang telah ditetapkan pada Bab III. Pengujian dilakukan berdasarkan skenario pengujian yang telah disusun sebelumnya, dengan fokus pada pengujian fungsional dan perilaku sistem frontend dalam mengelola data serta merespons interaksi pengguna.

Pada Bab ini, hasil pengujian disajikan dalam bentuk tabel hasil pengujian yang memuat hasil aktual dari setiap skenario pengujian. Informasi mengenai data uji dan hasil yang diharapkan tidak ditampilkan kembali pada bagian ini karena telah dijelaskan secara rinci pada Bab III. Dengan demikian, penyajian hasil pengujian pada Bab ini difokuskan pada pengamatan terhadap keluaran sistem dan status keberhasilan pengujian, guna menghindari pengulangan informasi dan menekankan pada hasil empiris dari proses pengujian yang telah dilakukan.

Berikut adalah hasil pengujian yang dilakukan pada sistem:

\begin{packed_enum}
    \item Landing Page
      \begin{longtable}{|c|p{2.5cm}|p{4cm}|p{4cm}|c|}
      \caption{Hasil Pengujian Landing Page}
      \label{tab:hasil-pengujian-landing-page} \\

      \hline
      \textbf{ID} &
      \textbf{Fitur} &
      \textbf{Skenario Pengujian} &
      \textbf{Hasil Aktual} &
      \textbf{Status} \\
      \hline
      \endfirsthead

      \hline
      \textbf{ID} &
      \textbf{Fitur} &
      \textbf{Skenario Pengujian} &
      \textbf{Hasil Aktual} &
      \textbf{Status} \\
      \hline
      \endhead

      \endfoot

      \hline
      \endlastfoot

      TC-LP-01 &
      Landing Page &
      Pengguna membuka aplikasi tanpa melakukan proses login &
      Halaman landing berhasil ditampilkan dengan informasi sistem serta navigasi menuju halaman login dan registrasi. &
      Berhasil \\
      \hline

      \end{longtable}

    \item Login
    
    \begin{longtable}{|c|p{2.5cm}|p{4cm}|p{4.5cm}|c|}
    \caption{Hasil Pengujian Fitur Login}
    \label{tab:hasil-pengujian-login} \\

    \hline
    \textbf{ID} &
    \textbf{Fitur} &
    \textbf{Skenario Pengujian} &
    \textbf{Hasil Aktual} &
    \textbf{Status} \\
    \hline
    \endfirsthead

    \hline
    \textbf{ID} &
    \textbf{Fitur} &
    \textbf{Skenario Pengujian} &
    \textbf{Hasil Aktual} &
    \textbf{Status} \\
    \hline
    \endhead

    \endfoot

    \hline
    \endlastfoot

    TC-LG-01 &
    Login &
    Pengguna melakukan login dengan kredensial yang valid &
    Sistem berhasil memverifikasi kredensial pengguna dan mengarahkan pengguna ke halaman dashboard. &
    Berhasil \\
    \hline

    TC-LG-02 &
    Login &
    Pengguna melakukan login dengan kata sandi yang salah &
    Sistem menampilkan pesan kesalahan autentikasi dan pengguna tetap berada pada halaman login. &
    Berhasil \\
    \hline

    TC-LG-03 &
    Login &
    Pengguna melakukan login dengan email yang tidak terdaftar &
    Sistem menampilkan pesan bahwa akun tidak ditemukan dan tidak melanjutkan proses login. &
    Berhasil \\
    \hline

    TC-LG-04 &
    Login &
    Pengguna mengirimkan formulir login dengan field kosong &
    Sistem menampilkan pesan validasi bahwa seluruh field wajib diisi sebelum proses login dilakukan. &
    Berhasil \\
    \hline

    TC-LG-05 &
    Login &
    Pengguna berhasil login setelah sebelumnya gagal login &
    Sistem menerima kredensial yang valid dan mengarahkan pengguna ke halaman dashboard tanpa kendala. &
    Berhasil \\
    \hline

    \end{longtable}


    \item Register
    \begin{longtable}{|c|p{2.5cm}|p{4cm}|p{4.5cm}|c|}
    \caption{Hasil Pengujian Fitur Register}
    \label{tab:hasil-pengujian-register} \\

    \hline
    \textbf{ID} &
    \textbf{Fitur} &
    \textbf{Skenario Pengujian} &
    \textbf{Hasil Aktual} &
    \textbf{Status} \\
    \hline
    \endfirsthead

    \hline
    \textbf{ID} &
    \textbf{Fitur} &
    \textbf{Skenario Pengujian} &
    \textbf{Hasil Aktual} &
    \textbf{Status} \\
    \hline
    \endhead

    \endfoot

    \hline
    \endlastfoot

    TC-RG-01 &
    Register &
    Pengguna melakukan pendaftaran dengan data yang valid &
    Sistem berhasil menyimpan data pengguna dan akun dapat digunakan untuk melakukan proses login. &
    Berhasil \\
    \hline

    TC-RG-02 &
    Register &
    Pengguna melakukan pendaftaran dengan data tidak lengkap &
    Sistem menampilkan pesan validasi bahwa seluruh data registrasi wajib diisi dan proses pendaftaran tidak dilanjutkan. &
    Berhasil \\
    \hline

    TC-RG-03 &
    Register &
    Pengguna melakukan pendaftaran dengan format data tidak sesuai &
    Sistem menampilkan pesan kesalahan terkait format data yang tidak valid dan meminta pengguna memperbaiki input. &
    Berhasil \\
    \hline

    TC-RG-04 &
    Register &
    Pengguna mendaftarkan akun dengan email yang sudah terdaftar &
    Sistem menampilkan pesan bahwa akun dengan email tersebut sudah terdaftar dan proses pendaftaran dibatalkan. &
    Berhasil \\
    \hline

    TC-RG-05 &
    Register &
    Pengguna mengirimkan ulang data registrasi setelah terjadi kesalahan &
    Sistem menerima data registrasi yang valid dan proses pendaftaran berhasil diselesaikan. &
    Berhasil \\
    \hline

    \end{longtable}

    \item Dashboard
    
    \begin{longtable}{|c|p{2.5cm}|p{4cm}|p{4.5cm}|c|}
    \caption{Hasil Pengujian Fitur Dashboard}
    \label{tab:hasil-pengujian-dashboard} \\

    \hline
    \textbf{ID} &
    \textbf{Fitur} &
    \textbf{Skenario Pengujian} &
    \textbf{Hasil Aktual} &
    \textbf{Status} \\
    \hline
    \endfirsthead

    \hline
    \textbf{ID} &
    \textbf{Fitur} &
    \textbf{Skenario Pengujian} &
    \textbf{Hasil Aktual} &
    \textbf{Status} \\
    \hline
    \endhead

    \endfoot

    \hline
    \endlastfoot

    TC-DB-01 &
    Dashboard &
    Pengguna membuka halaman dashboard setelah login &
    Sistem berhasil menampilkan ringkasan data dan visualisasi utama dashboard sesuai data yang tersedia. &
    Berhasil \\
    \hline

    TC-DB-02 &
    Dashboard &
    Pengguna membuka dashboard saat data belum tersedia &
    Sistem menampilkan indikator pemuatan dan informasi status tanpa menimbulkan kesalahan tampilan. &
    Berhasil \\
    \hline

    TC-DB-03 &
    Dashboard &
    Pengguna melakukan navigasi ke menu lain dan kembali ke dashboard &
    Data dashboard tetap ditampilkan secara konsisten tanpa kehilangan atau perubahan data. &
    Berhasil \\
    \hline

    TC-DB-04 &
    Dashboard &
    Pembaruan data dashboard dari server &
    Sistem memperbarui tampilan dashboard sesuai dengan data terbaru yang diterima. &
    Berhasil \\
    \hline

    TC-DB-05 &
    Dashboard &
    Beberapa komponen menggunakan sumber data yang sama &
    Seluruh komponen dashboard menampilkan data yang konsisten karena mengacu pada satu sumber data yang dipercaya. &
    Berhasil \\
    \hline

    \end{longtable}


    \item Sentiment
    
    \begin{longtable}{|c|p{2.5cm}|p{4cm}|p{4.5cm}|c|}
    \caption{Hasil Pengujian Halaman Dashboard Sentiment}
    \label{tab:hasil-pengujian-sentiment} \\

    \hline
    \textbf{ID} &
    \textbf{Fitur} &
    \textbf{Skenario Pengujian} &
    \textbf{Hasil Aktual} &
    \textbf{Status} \\
    \hline
    \endfirsthead

    \hline
    \textbf{ID} &
    \textbf{Fitur} &
    \textbf{Skenario Pengujian} &
    \textbf{Hasil Aktual} &
    \textbf{Status} \\
    \hline
    \endhead

    \endfoot

    \hline
    \endlastfoot

    TC-ST-01 &
    Dashboard Sentiment &
    Pengguna membuka halaman dashboard sentiment &
    Sistem berhasil menampilkan visualisasi dan ringkasan hasil analisis sentimen sesuai dengan data yang tersedia. &
    Berhasil \\
    \hline

    TC-ST-02 &
    Dashboard Sentiment &
    Pengguna membuka halaman sentiment saat data belum tersedia &
    Sistem menampilkan indikator pemuatan atau pesan informasi tanpa menimbulkan kesalahan tampilan. &
    Berhasil \\
    \hline

    TC-ST-03 &
    Dashboard Sentiment &
    Pembaruan data sentimen dari server &
    Sistem memperbarui visualisasi dan ringkasan sentimen sesuai dengan data sentimen terbaru. &
    Berhasil \\
    \hline

    TC-ST-04 &
    Dashboard Sentiment &
    Navigasi ke halaman lain dan kembali ke dashboard sentiment &
    Data sentimen tetap ditampilkan secara konsisten tanpa kehilangan atau perubahan data. &
    Berhasil \\
    \hline

    TC-ST-05 &
    Dashboard Sentiment &
    Beberapa komponen menampilkan data sentimen yang sama &
    Seluruh komponen visualisasi menampilkan data sentimen yang konsisten karena mengacu pada satu sumber data yang dipercaya. &
    Berhasil \\
    \hline

    \end{longtable}



    \item Recomendation Content
    
    \begin{longtable}{|c|p{2.5cm}|p{4cm}|p{4.5cm}|c|}
    \caption{Hasil Pengujian Halaman Dashboard Rekomendasi Konten}
    \label{tab:hasil-pengujian-recommendation} \\

    \hline
    \textbf{ID} &
    \textbf{Fitur} &
    \textbf{Skenario Pengujian} &
    \textbf{Hasil Aktual} &
    \textbf{Status} \\
    \hline
    \endfirsthead

    \hline
    \textbf{ID} &
    \textbf{Fitur} &
    \textbf{Skenario Pengujian} &
    \textbf{Hasil Aktual} &
    \textbf{Status} \\
    \hline
    \endhead

    \endfoot

    \hline
    \endlastfoot

    TC-RC-01 &
    Dashboard Rekomendasi &
    Pengguna membuka halaman dashboard rekomendasi konten &
    Sistem berhasil menampilkan daftar rekomendasi konten berdasarkan data analisis sentimen yang tersedia. &
    Berhasil \\
    \hline

    TC-RC-02 &
    Dashboard Rekomendasi &
    Pengguna membuka halaman rekomendasi saat data belum tersedia &
    Sistem menampilkan indikator pemuatan atau pesan informasi tanpa menimbulkan kesalahan tampilan. &
    Berhasil \\
    \hline

    TC-RC-03 &
    Dashboard Rekomendasi &
    Pembaruan data rekomendasi dari server &
    Sistem memperbarui daftar rekomendasi sesuai dengan data rekomendasi terbaru yang diterima. &
    Berhasil \\
    \hline

    TC-RC-04 &
    Dashboard Rekomendasi &
    Navigasi ke halaman lain dan kembali ke dashboard rekomendasi &
    Data rekomendasi tetap ditampilkan secara konsisten tanpa kehilangan atau perubahan data. &
    Berhasil \\
    \hline

    TC-RC-05 &
    Dashboard Rekomendasi &
    Konsistensi rekomendasi dengan data sentimen &
    Rekomendasi konten menyesuaikan perubahan data sentimen yang digunakan sebagai dasar penyusunan rekomendasi. &
    Berhasil \\
    \hline

    \end{longtable}


    \item Scraper
    
    \begin{longtable}{|c|p{2.5cm}|p{4cm}|p{4.5cm}|c|}
    \caption{Hasil Pengujian Halaman Data Scraper}
    \label{tab:hasil-pengujian-scraper} \\

    \hline
    \textbf{ID} &
    \textbf{Fitur} &
    \textbf{Skenario Pengujian} &
    \textbf{Hasil Aktual} &
    \textbf{Status} \\
    \hline
    \endfirsthead

    \hline
    \textbf{ID} &
    \textbf{Fitur} &
    \textbf{Skenario Pengujian} &
    \textbf{Hasil Aktual} &
    \textbf{Status} \\
    \hline
    \endhead

    \endfoot

    \hline
    \endlastfoot

    TC-SC-01 &
    Data Scraper &
    Pengguna membuka halaman data scraper &
    Sistem berhasil menampilkan daftar data hasil scraping yang diperoleh dari server. &
    Berhasil \\
    \hline

    TC-SC-02 &
    Data Scraper &
    Pengguna membuka halaman scraper saat data belum tersedia &
    Sistem menampilkan indikator pemuatan atau pesan informasi tanpa menimbulkan kesalahan tampilan. &
    Berhasil \\
    \hline

    TC-SC-03 &
    Data Scraper &
    Pengguna memilih data untuk dianalisis &
    Sistem berhasil menandai dataset yang dipilih dan data tersebut siap digunakan pada proses analisis. &
    Berhasil \\
    \hline

    TC-SC-04 &
    Data Scraper &
    Navigasi ke halaman lain dan kembali ke data scraper &
    Data hasil scraping tetap ditampilkan secara konsisten tanpa kehilangan atau perubahan data. &
    Berhasil \\
    \hline

    \end{longtable}

      
  \end{packed_enum}


\section{Pembahasan}


Analisis kinerja sistem dilakukan untuk mengevaluasi efektivitas dan efisiensi sistem dalam mencapai tujuan yang telah ditetapkan. Bagian ini berfokus pada pengukuran dan analisis terhadap parameter-parameter utama yang mencerminkan kualitas kinerja sistem.
